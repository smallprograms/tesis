A grid infrastructure is composed by several resources, which are
heterogeneous, distributed geographically, and may belong to diverses
institutions. These resources must interact between them and with
other grid components. In this scenario, the {\itshape opennes} of the
grid is a difficult issue to achieve.

In distributed systems, opennes is the property such that each
component is continually open to interaction with other systems. In a
grid environment, grid resources should provide, at any point of time,
new services, functions, or in general, new resources that are unknown
a priori to the grid users or other grid components. Having an
standarized manner to describe resources, this task could be more
simple. 

Existing resource description is highly constrained, it does not
provides proper mechanisms that allows to extend the schemas of
description when new elements are incorporated. Hence, {\itshape
metadata} --data about data-- plays a fundamental rol, together the use of {\itshape
ontology} for resource description in the field of {\itshape web
semantic}.%  helping to grid resources to know
%about other grid resources, thus, the use of an ontology language is
%fundamental to provide a semantic description of these resources

Web Semantic has as key idea to explicate the meaning of web content
by adding semantic notations, allowing the vast range of available
information and services to be more effectively exploited.

This chapter describes some important concepts related with the
Semantic Web and a new proposed framework, which provides the
capability of use semantic web in the ATLAS Grid Information System in
order to provide semantic resource descritpion, thus, providing a more
rich description of the resources, services and topology of the ATLAS
grid.

\section{Semantic Web}
%El termino semantica se refiere a los aspectos del significado o interpretacion
%del significado de un determinado simbolo, palabra, lenguage o representacion
%formal
% RAE, Semantica
% 1.  adj. Perteneciente o relativo a la significaci�n de las palabras.
% 2.  f. Estudio del significado de los signos ling��sticos y de sus combinaciones, desde un punto de vista sincr�nico o diacr�nico.
% Significacion: Sentido de una palabra o frase

The Semantic Web aims at adding semantics to the data published on the
Web, so that machines are able to process these data in a similar way
a human can do, allowing {\itshape knowledge representation}. 

The semantic representation of the data allows the machines have
access to structured collections of informations and sets of inference
rules that they can use to conduct automated reasoning, and the
{\itshape ontologies}, are the backbone technology
\cite{bib:Berners-Lee}.

%An important concept is {\itshape metadata}, which is data about data
%and in the grid environments, metadata helps to grid resources to know
%about other grid resources, thus, the use of an ontology language is
%fundamental to provide a semantic description of these resources.

There is not a universal definition of ontology, 
%The term \textbf{ontology} has two different views, the first one, for
%philosophers, where {\itshape Ontology} referes to the ``essence of
%things through the changes''\cite{webSemCardoso}, and the second one,
%on the computer science side, the most frequently quoted definition of
but in the computer science field, the most frequently quoted
definition of {\itshape ontology} is Gruber's\footnote{Tom Gruber is a
researcher with a focus on systems for knowledge sharing and
collective intelligence
\url{http://tomgruber.org/bio/short-bio.htm}}:\\

\
{\itshape `` An ontology is a formal, explicit specification of a shared
conceptualization.''}
\\

From this definition:

\begin{itemize}
\item {\itshape formal} indicates that the specification should be machine processable
\item {\itshape explicit} means that the elements must be clearly defined 
\item {\itshape conceptualization} stands for an abstract model
\end{itemize}


Thus, in Gruber's view, an ontology is the representation of the
knowledge of a domain, where a set of objects and their relationships
is described by a vocabulary \cite{webSemCasanova}. 

Another defintion, according to Sowa\footnote{John F. Sowa is a
researcher cofounder of VivoMind
Intelligence. \url{http://www.jfsowa.com/pubs/}} is that an ontology
is:\\

{\itshape ``The study of categories of things that exist or may exist
in some domain. The product of such study, called an ontology, is a
catalogye of types of things that are assumed to exist in a domain of
interest \textbf{D} from the perspective of a person who usses a
language \textbf{L} for the purpose of talking about \textbf{D}.}\\


There are {\itshape ontology description languages} to define
ontologies, such as the Resource Description Framework (RDF), RDF
Schema (RDF-S), and the Web Ontology Language (OWL).


In the following sections we give an overview of the aforementioned
ontology description languages.

%The ATLAS Grid Information System provides information about the
%resources, services and topology of the ATLAS grid.


\section{RDF}
\label{sec:rdf}

The {\itshape Resource Description Framework (RDF)} is a
general-purpose language for representing information about resources
in the Web. It is particularly intended for representing metadata
about Web resources, but it can also be used to represent information
about ojbects that can be identified on the Web.

To understand RDF structure, firstly it is necessary to have clear
some concepts of the Extensible Markup Language (XML), which are
describe as follows.

\subsection{XML Concepts}
XML is a markup language that allows to exchange of a wide variety of
data on the Web and elsewhere. The XML concepts described as follows,
play an important role in the ontology languages.

\begin{description}
\item[Resource:] 
is anything that has an identity, be it a retrievable digital entity
(such an electronic document, an image or a service), a physical
entity (such as a book) or a collection of other resources.

\item[URI:] 
the Uniform Resource Identifier is a character string that identifies
an abastract or physical resource on the Web, for instance,
\url{http://atlas.cern.ch/0/Site1.html}

\item[URIref:] 
the URI reference denotes the common usage of a URI, with an optional
fragment identifier attached to it and preceded by the character
``\#''. However, the URI that results from such a reference includes
only the URI after removing the fragment identifier, for instance:
\url{http://atlas.cern.ch/0/topology#cloud}.

An {\itshape absolute URIref} identifies a resource independently of
the context in which the URIref appears. A {\itshape relative URIref}
is a URIref with some prefix ommited, hence, information from the
context in which the URIref appears is required to fill in the omitted
prefix
%In particular, a relative URIref consisting of just a fragment identifier 
%is equivalent to the UURIref of the document in which it apperars, 
%with the fragment identifier appended to it. For example, 
%the relative UIRIref #cloud, appearing in a document identified 
%by the URIref \url{http://atlas.cern.ch/0/topology} is considered 
%equivalent to the URIref: \url{http://atlas.cern.ch/0/topology#cloud}

\item[Namespace:] 
is a collection of names, which is identified by a URIref. For
instance the RDF namespace has the URIref
\url{http://www.w3.org/1999/02/22-rdf/syntax/ns#} and the prefix
\url{rdf}.

\begin{table}[h!tpb]
\begin{tabular}{|l|l|l|}\hline
\textbf{Namespace} &\textbf{URIref} &\textbf{Prefix} \\\hline
RDF & \url{http://www.w3.org/1999/02/22-rdf/syntax/ns#} & rdf \\\hline
Dublin Core &\url{http://purl.org/dc/elements/1.1/} & dc \\\hline
\end{tabular}
\caption{Examples of namespaces}
\label{tab:namespaces}
\end{table}
\item[Qualified Name:]
Names from namespaces can appear as qualified names (QNames) of the
form $P:L$, containing a single colon ``:'', that separates the name
into namespace prefix $P$ and a local part $L$. The namespace prefix
must be associated with a namespace URIref $N$ in a namespace
declaration. The qualified name represents the absolute URIref
constructed by concatenating $N$ and $L$. For instance, the Qname
\url{rdf:description} ($P:L$):

\begin{itemize}
\item 
it has namespace prefix \url{rdf} ($P$)
\item  
it has local part \url{description} ($L$)
\item
its namespace prefix is associated to the namespace URIref
\url{http://www.w3.org/1999/02/22-rdf-syntax-ns\#} ($N$)
\item  
it expands the URIref
\url{http://www.w3.org/1999/02/22-rdf-syntax-ns\#description} 
(concatenation of $N$ and $L$)
\end{itemize}

\end{description}

A namespace is declared using a family of reserved attributes, whose
name must either be \url{xmlns} or have \url{xmlns:} as a prefix, and
whose value is a URIref identifying the namespace, for instance:

\begin{figure}[h!tpb]
\begin{verbatim}
<rdf:RDF
    xmlns:atlas="http://atlas.cern.ch/0/"
    xmlns:rdf="http://www.w3.org/1999/02/22-rdf-syntax-ns#">
\end{verbatim}
\caption{Namespace declaration.}
\end{figure}



%The XML is a general-purpose markup language, designed to describe
%structured documents.

%An XML document consist of plain text and markup, in the form of tags,
%which is interpretated by applications programs:

%\begin{verbatim}
%XML
%\end{verbatim}

\subsection{RDF Statements and Vocabulary}
RDF has a very simple and flexible data model, based on the central
concept of the RDF statement. RDF also consider the concept of
vocabulary, due to its relevance to ontology modeling.


An \textbf {RDF statement} is a triple $(S,P,O)$, where:

\begin{itemize}
\item 
$S$ is the {\itshape subject} and correspond to  a URIref

\item 
$P$ is the {\itshape property} and corresponds to a URIref that
denotes a binary relationship

\item 
$O$ is the object and corresponds either a URIref or a literal. If $O$
is a literal, then $O$ is also called the value of the property $P$

\end{itemize}


An example of an RDF statement is shown if
Tab. \ref{tab:RDFStatements}\\

\begin{table}[h!tpb]
\begin{tabular}{|l|l|l|}\hline
\textbf{Element} &\textbf{Value (Absolute URIref of Literal)} &\textbf{Value (QName)} \\\hline
Subject & \url{http://atlas.cern.ch/0/#Site1} & \\\hline
Property & \url{http://atlas.cern.ch/0/ispartof} & \url{atlas:ispartof} \\\hline
Value &\url{http://http://atlas.cern.ch/0/#Cloud0} & \\\hline
\end{tabular}
\caption{RDF statement.}
\label{tab:RDFStatements}
\end{table}

A \textbf{{\itshape vocabulary}} is a set of URIrefs and is therefore
synonymous with an XML namespace. A vocabulary $V$ is frequently
specified in two alternatives way: the first one uses qualified names
to define $V$ and the second one uses fragment identifiers to define
$V$.


\begin{description}
\item[Using Qualified Names:]
Defines $V$ selecting a fixed URIref $U$ and a prefix $p$ for it, then
defines a set of qualified names with prefix $p$ and finally defines
$V$ as a set of URIrefs represented by such qualified names.

\item[Using Fragment Identifiers:]
Defines $V$ selecting a fixed URIref $U$ and a prefix $p$ for it, then
define a set of fragment identifiers and finally, define $V$ as the
set of URIrefs obtained by concatenating $U$, the charactyer
``$\#$''and each fragment identifier.
\end{description}


\subsection{RDF Notations}
RDF offers thee equivalent notations: RDF triples, RDF graphs and
RDF/XML, which are described as follows:


The \textbf{RDF triple} for an RDF statement $(S,P,O)$ is a string of
one of the two forms:

$<S> <P> <O> . $, if O is an absolute or relative URIref

$<S>  <P> ``O''$, if O is a literal

\textcolor{red}{ARREGLAR!}

%\newpage
For instance, RDF triples for statements in
Tab. \ref{tab:RDFStatements} are shown in Fig \ref{fig:RDFTriples}.

\begin{figure}[h!tpb]
\begin{verbatim}
<http://atlas.cern.ch/0/#Site>
    <http://atlas.cern.ch/0/ispartof>
        <http://http://atlas.cern.ch/0/#Cloud> .

<http://atlas.cern.ch/0/#Site>
    <http://atlas.cern.ch/0/contact>
        ``Richard Brooks'' .
\end{verbatim}
\caption{RDF Triples.}
\label{fig:RDFTriples}
\end{figure}

or, using qualified names for the URIrefs in the dc vocabulary:

\begin{figure}[h!tpb]
\begin{verbatim}
<http://atlas.cern.ch/0/#Site>
    atlas:ispartof <http://http://atlas.cern.ch/0/#Cloud> .

<http://atlas.cern.ch/0/#Site>
    atlas:contact ``Richard Brooks'' .
\end{verbatim}
\caption{RDF Triples using Qnames.}
\end{figure}

The \textbf{RDF graph} notation translates a set of RDF statements
into a graph (Fig. \ref{fig:RDFgraph}), where the subjects and the
objects are the nodes, and the properties are represented by the arcs.


\begin{figure}[!htbp]
\centering
\pgfuseimage{RDFgraph}
%\epsfig{file=images/RDFgraph.eps,scale=1.0}
%\includegraphics[width=\textwidth]{RDFgraph.eps}
\caption{RDF graph for a RDF triple.}
\label{fig:RDFgraph}
\end{figure}


\newpage
The \textbf{RDF/XML} is the notation for RDF statements preferred in
the context of software agents exchanging data in open distributed
environments. This notations encode the RDF graph in XML. The nodes
and predicates are represented in XML terms, using element names,
attribute names, element contents and attribute values.

\begin{figure}[h!tpb]
\begin{verbatim}
<?xml version="1.0">
<rdf:RDF
    xmlns:rdf="http://www.w3.org/1999/02/22-rdf-syntax-ns#"
    xmlns:atlas="http://atlas.cern.ch/0">
    <rdf:Description 
        rdf:about="http://atlas.cern.ch/0/Site1">
	        <atlas:ispartof
	            rdf:resource="http://atlas.cern.ch/0/CloudA"/>
    </rdf:Description>
</rdf:RDF> 
\end{verbatim}
\caption{RDF/XML Example.}
\label{fig:RDFXML}
\end{figure}

A property may itself be a resource, for instance the Site1 has a
contact Richard Brooks, but the contact can be represented by a
contact resource, which has properties name and email. An XML
representation of this would be which is shown if
Fig. \ref{fig:RDFXML2}.

\begin{figure}[h!tpb]
\begin{verbatim}
<?xml version="1.0">
<rdf:RDF
    xmlns:rdf="http://www.w3.org/1999/02/22-rdf-syntax-ns#"
    xmlns:atlas="http://atlas.cern.ch/0">
    <rdf:Description 
        rdf:about="http://atlas.cern.ch/0/Site1">
		<atlas:contact
		 rdf:resource="http://atlas.cern.ch/0/contact"/>
    </rdf:Description>
    <rdf:Description
        rdf:about="http://atlas.cern.ch/0/contact">
	<atlas:email>richard@cern.ch"<atlas:email/>
	<atlas:name>Richard Brooks"<atlas:name/>	
    </rdf:Description>	   
</rdf:RDF> 

or

<?xml version="1.0">
<rdf:RDF
    xmlns:rdf="http://www.w3.org/1999/02/22-rdf-syntax-ns#"
    xmlns:atlas="http://atlas.cern.ch/0">
    <rdf:Description 
        rdf:about="http://atlas.cern.ch/0/Site1">
    		<atlas:contact>
			    <rdf:Description rdf:about="http://atlas.cern.ch/0/contact">
				     <atlas:email>richard@cern.ch"</atlas:email>
				     <atlas:name>Richard Brooks"</atlas:name>
			    </rdf:Description>  
		</atlas:contact>
    </rdf:Description>     
</rdf:RDF>


\end{verbatim}
\caption{RDF/XML Example 2.}
\label{fig:RDFXML2}
\end{figure}


\newpage
The qualified name {\itshape rdf:type} of the RDF vocabulary is a
built-in property with a predefined semantics. %An RDF statement of the
%form ($S$, {\itshape rdf:type}, $O$) indicates that resource $O$
%represents a category or a class of resources, of which resource $S$
%is an instance. Those resources are called {\itshape typed node
%elements} in RDF/XML documents.
This property indicates that a resource belong to a set of resources,
or a class. Both, the subject and object of the statement using
rdf:type property must be a resource (not a literal), and they are
called {\itshape typed node elements} in RDF/XML documents. For
instance, we can add and {\itshape rdf:type} property to the example
of Fig. \ref{fig:RDFXML} to indicate that a site belongs to a cloud in
the ATLAS grid.

\begin{figure}[h!tpb]
\begin{verbatim}
<?xml version=''1.0''>
<rdf:RDF xml:base="http://atlas.cern.ch/0"
    xmlns:rdf="http://www.w3.org/1999/02/22-rdf-syntax-ns#"
    xmlns:dc="http://http:purl.org/dc/elements/1.1/"
    <rdf:Description
        rdf:about="http://atlas.cern.ch/0/Site1">
	        <rdf:type rdf:resource="http://atlas.cern.ch/0/topology/Entity"/>
	                <atlas:ispartof
	                  rdf:resource="http://atlas.cern.ch/0/CloudA"/>
    </rdf:Description>
</rdf:RDF> 
\end{verbatim}
\caption{rdf:type example.}
\label{fig:RDFtypeXML}
\end{figure}


\section{RDF Schema}
RDF Schema (or RDF-S) can be described as a semantic extension of
RDF. RDF provides enormous flexibility but, it provides no means for
defining applications-specific classes and properties. For instance,
while RDF can represent that contact is a property of the Site1, there
is no standard vocabulary for stating that contact is an attribute
which applises to all sites. To overcome this lack, the RDF Schema
provides the classes and properties.

RDF Schema is defined as a vocabulary, whose URIref is
\url{http://www.w3.org/2000/01/ref-schema#} and whose name is
{\itshape rdfs}.


\subsection{Classes}
Resources may be divided into groups called classes. The members of a
class are known as instances of the class. Classes are themselves
resources. They are often identified by RDF URI References and may be
described using RDF properties. 

In RDF Schema a {\itshape class} is any resource having an {\itshape
rdf:type} property whose value is the qualified name {\itshape
rdfs:Class} of the RDF Schema vocabulary.

A {\itshape subclass} is a transitive relationship between two
classes, where is used the predifined {\itshape rdfs:subClassOf}
property to relate the clases.

Using RDF Schema, we can represent that "contact is an attribute that
applies to all sites". The site class could be represented as a
subclass of Resource:


\begin{verbatim}
<?xml version=''1.0''>
<rdf:RDF xml:base="http://atlas.cern.ch/0"
    xmlns:rdf="http://www.w3.org/1999/02/22-rdf-syntax-ns#"
    xmlns:dc="http://http:purl.org/dc/elements/1.1/"
    <rdf:Description
        rdf:about="http://atlas.cern.ch/0/Site">
	        <rdf:type rdf:resource="http://www.w3.org/2000/01/rdf-schema#Class"/>
	                <rdfs:subClassOf
			    rdf:resource="http://www.w3.org/2000/01/rdf-schema#Resource"/>
    </rdf:Description>
</rdf:RDF> 
\end{verbatim}

And the contact property can be represented as a property which has Site as its domain and


\begin{verbatim}
<?xml version=''1.0''>
<rdf:RDF xml:base="http://atlas.cern.ch/0"
    xmlns:rdf="http://www.w3.org/1999/02/22-rdf-syntax-ns#"
    xmlns:rdf="http://www.w3.org/2000/01/rdf-schema#"
    xmlns:dc="http://http:purl.org/dc/elements/1.1/"
    
            <rdfs:Class rdf:about="http://atlas.cern.ch/0/Site"/>
</rdf:RDF> 
\end{verbatim}




\subsection{Properties}
A property is any instance of the class {\itshape rdfs:Property}. The
{\itshape rdf:domain} property is used to indicate that a particular
property applies to a designated class, and the {\itshape rdf:range}
property is used to indicate that the values of a particular property
are instances of a designated class or are instances (literals) of an
XML Schema datatype.

The relation between two properties is described using the predefined
{\itshape rdfs:subPropertyOf} subproperty.

\begin{verbatim}
<?xml version=''1.0''>
<rdf:RDF xml:base="http://atlas.cern.ch/0"
    xmlns:rdf="http://www.w3.org/1999/02/22-rdf-syntax-ns#"
    xmlns:rdf="http://www.w3.org/2000/01/rdf-schema#"
    xmlns:dc="http://http:purl.org/dc/elements/1.1/"
    
            <rdfs:Class rdf:about="Site"/>
	     <rdf:Property rdf:about="noServices"/>
	             <rdfs:domain rdf:resource="Site"/>
		     <rdfs:range rdf:resource="&xsd;positiveInteger"/>
	     </rdf:Property>
</rdf:RDF> 
\end{verbatim}

\section{OWL}

%In the Semantic Web literature there is not an universal definition of
%ontology\footnote{Ontology comes from the Greek {\itshape ontos}
%(being) + {\itshape logos} (word). This term was introduced in
%philosophy, in the nineteenth centtury by German philosophers, to
%distinguish the studie of the being from the study of various kinds of
%beings in the natural sciences.}, but a proper one is that ontology is
%the representation of the knowledge of a domain, where a set of
%objects and their relationships are described by a vocabulary.

%Ontologies are used to capture knowledge about some domain of
%interest. An ontology describes the concepts the domain and also the
%relationships that hold between those concepts.

OWL is an standard ontology language that describes classes,
properties and relations among these conceptual objects in a way that
facilitates machine interpretability of Web content. OWL is defined as
a vocabulary, just as are RDF and RDF Schema, but it has a richer
semantics \cite{webSemCasanova}. OWL provides a set of operators, such
as $and$, $or$ or $negation$. Complex concepts can therefore be built up in
definitions out of simple concepts.

This ontology language can be categorised into three sub-languages:
OWL-Lite, OWL-DL and OWL-Full. A defining feature of each sub-language
is its expressiveness. The expresiveness of OWL-DL is between OWL-Lite
and OWL-Full, thus, OWL-DL can be considered as an extension of
OWL-Lite and OWL-Full as an extension of of OWL-DL. 
%from protege tutorial

\begin{description}
\item[OWL-Lite]
is intended to be used when only a simple class hierarchy and simple
constraints are needed, with enough power to model {\itshape
thesauri}\footnote{} and simple ontologies.

\item[OWL-DL] 
is more expressive than OWL-Lite and offers all OWL constructs, under
certain limitations.

\item[OWL-Full] 
is intended to be used in situations where very high expresiveness is
more important than being able to guarantee the decidability or
computational completness of the language. It is therefore not
possible to perform automated reasoning on OWL-Full ontologies.
%protege tutorial

\end{description}



%\section{RDF in ATLAS Grid Information System}



%``the stored ATLAS topology should be available in that format for
%consumption by external tools (especially the rdf store)''

%\begin{itemize} 

%\item 
%Create metadata (data about data) for the ATLAS topology, in order to
%provide RDF ATLAS topology to external components.

%\item 
%RDF and OWL allows to create the metadata, representing the resources
%an the relations between them.


%\item 
%Defining classes, to group together related resources in order to
%create ontology: computational resources, storages resources, network
%resources


%\item 
%Currently lack in the ontological description of the grid services

%\item  
%TODO: A look to current model definitions (specifications and schema):
%CIM

%\item 
%Doubts: propose the RDF schema for ATLAS topology, use existing tools
%to modelate,


%\item 
%RDF will provide OPENNES and STANDARIZATION to the grid system providing information to be processed by machines, inforamtion in a RDF format, using ontology to create vocabularies and rules ... to discuss the innovation


%\end{itemize}



%\section{Proposed RDF Component to the ATLAS Grid Information System}

\section{Semantic Resource Description for the \-ATLAS\ Grid}
The openness in a distributed system is fundamental to achieve an
efficient interaction of the components. In a grid environment, any
component should be able to provide, at any point of time, new
services, functions, or, in general, new resources that are unknown a
priori to its clients or other grid components. To incorporate and
interact with these new components, opennes and standarization are
important points. Hence, metadata plays a fundamental rol. An
usual representation of metada is the use of vocabularies that are
defined and agreed by communities, to ensure some degree of
interoperability across the applications and/or middleware that use
this metadata.

Metadata can be represented using the Web Semantic paradigm, through
the use of ontologies, which represents the data in structured
collection and set of inference rules.


The ATLAS Grid Information System provides a sub-set of information
--only static and semi-static-- related with resources, services and
topology of the ATLAS grid. If we extend this sub-set considering
static and dynamic information, we are in a more complex scenario,
where the volume of data increase considerable (jobs thar are running
in all the sites of the grids, storage capacity used in each tier,
cloud, or site, etc.)


\section{AGIS Semantic Architecture}
As was mentioned before, the {\itshape openness} in a distributed
system is fundamental to achieve an efficient interaction of the
components in a distributed system.  The AGIS architecture defined in
Sec. \ref{sec:agisArch} provides static and semi-static information
related with the resources, services and topology of the ATLAS grid.
The whole set of data, should include the dynamic information, which
currently is managed by the Monitoring area of the ATLAS grid.
%in diverse
%formats (XML, JSON, CSV) and uses a relational database to store the
%information collected by the agents.


Now, we propose a new architecture, wich uses the previous AGIS
architecture, but adds new components, to form the the {\itshape AGIS
Semantic Framework}, which allows to provide the information in a RDF
format, using the web semantic paradigm to the generation of the data.

The main idea of use Web Semantic in the ATLAS Grid Information 


This framework adds three new components that communicate, process and
exchange information. The first component process the information and
comunicates with the agents to processing the data and generate the
RDF format, the second one is the RDF storage, and the third one is
the RDF connector wich allows the communication between the clients
and the system itself.

Figure \ref{fig:RDFAGISArchitecture} depicts the new framework
proposed.

\begin{figure}[!htbp]
\centering
\pgfuseimage{RDFAGISArchitecture}
%\epsfig{file=images/RDFAGISArchitecture.eps,scale=0.55}
\caption{ATLAS Grid Information System Architecture.}
\label{fig:RDFAGISArchitecture}
\end{figure}

The new components are described as follows.

\subsection{RDF Data Processor}
The AGIS agents collect information from the different sources, and
then communicate with the DAO layer to store the information in a
relational database. To store information in a RDF format it is
necessary to process the collected data, hence, the {\itshape RDF Data
Processor} (RDP) component is in charge of this task.

Once the AGIS agents have retrieved the information, they establish
comunnication with The RDP component in order to process that
information to obtain the RDF schema.

The data processing done by the RDP component corresponds to the
generation of ontologies, which implies the conversion of the data
retreived by the AGIS agents into a RDF Model.

\begin{figure}
\pgfuseimage{RDPComponent}
%\epsfig{file=images/RDPComponent.eps,scale=0.55}
\caption{ATLAS Grid Information System Architecture.}
\label{fig:RDComponent}
\end{figure}

The RDF schema generation means that has to be a method that convert
the collected data into the classes and properties with all kinds of
Semantic Web vocabularies.


Firstly, it is necessary to have defined the RDF data model, that is,
the classes and properties that represent the ATLAS grid (sites,
services and topology). The RDF schema proposed will be based in the
{\itshape Core Grid Ontology} \cite{bib:coregrid}, which defines
fundamental grid domain concepts, but it is necessary to add new
concepts, vocabularies and relationships in order to reflect properly
the ATLAS grid.


The proposed grid model of Core Grid Ontology group the grid concepts
in three categories: VO-related, grid resource and grid middleware
service. The Fig. \ref{fig:coregrid} depicts the ontology classes
proposed by the Core Grid Ontology and the
Fig. \ref{fig:coregridATLAS} is the modification of the model, to
reflect and represent the ATLAS grid concepts.

\begin{figure}
\centering
\pgfuseimage{coregrid}
\caption{Core Grid Ontology Classes}
\label{fig:coregrid}
\end{figure}


\begin{figure}
\centering
%\pgfuseimage{coregridATLAS}
\caption{ATLAS Grid Ontology Classes}
\label{fig:coregridATLAS}
\end{figure}



%Fig. \ref{fig:RDFGeneration} shows the
%algorithm that performs that conversion.


%\begin{figure}[!htbp]
%\centering
%\begin{verbatim}
%WHILE not finished collected data
%    identify classes
%    bind classes to properties
%    write in RDF style
%END    
%\end{verbatim}
%\caption{RDF schema generation}
%\label{fig:RDFGeneration}
%\end{figure}

\subsection{RDF Storage}


\subsection{RDF Connector}





\section{Summary}