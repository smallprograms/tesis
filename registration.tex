One of the first fundamental steps to reconstruct a 3D scene is the 
positioning of the 3D points on a common coordinate system, conserving 
the original scene structure. This step is called registration. 

In general, when the sensor is collecting the 3D points from the scene, 
 its necessary to move or rotate it in order to capture new objects and surfaces from 
the scene, adding more information to the reconstructed scene. But in order to be 
able to infer the correct position of each capture, is necessary to have part 
of the view in common between two captures (an overlaping area). Thus is posible 
to position a new capture with respect to an old one in the scene, using this overlaping 
area to correctly align both captures.
 
The overlaping areas of the different captures of a scene must be correctly 
aligned when registering the points in a common coordinate system, 
thus with each new capture more information is added to the scene, 
getting closer to the desired result. 

The most used algorithm for solve this problem is called 
ICP (Iterative Closest Point) Algorithm. Given two point clouds, 
this algorithm find the best rigid transformation to align both clouds 
(minimize distance between puntos emparejados). In order to make this, 
the algorithm starts with an initial guess of the transformation. 
At each iteration the algorithm matchs points of each cloud and then 
find the best transformation. 




