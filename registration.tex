
The most used algorithm to register point clouds into a common coordinate 
system is called 
ICP (Iterative Closest Point) Algorithm. Given two point clouds, 
this algorithm find the best rigid transformation to align both clouds 
(minimize distance between closest points). In order to make this, 
the algorithm starts with an initial guess of the transformation. 
At each iteration the algorithm matchs closest points between the two point clouds  and then 
finds the rigid transformation that minimizes euclidian distance between closest points.

This algorithm has some drawbacks, such as beign susceptible to local minima, 
have small convergence basin, and in general needing a high number of iterations \cite{Rusu2009}. For this reason additional 
algorithms are used to compute a good approximation of the desired transform, in order to start the ICP iterations closer to the solution, also
 other techniques are used in combination with ICP, such as filtering, photoconsistency, graph optimization, etc. In order improve results and 
reduce the amount of computations required to find the transformation.


This proposal consists on modificated version of ICP in order to work with RGB-D data, combining the use of geometrical and visual clues in 
order to reduce the computational cost and improve the obtained results.
