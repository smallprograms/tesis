

This thesis provides a registration algorithm that uses geometrical and visual information 
to perform the alignment between RGB-D frames.  

A first estimation of the relative transformation between each pair of successive frames was obtained using 
visual information, with optical flow and SURF methods. An edge filtering technique was applied to detect a representative subset 
of the data, using only between 10\% and 20\% of original points as input to ICP algorithm. Finally a pose graph optimization method increases overall result quality.

The proposed algorithm reduces the execution time and takes advantage of rich textured surfaces to align point clouds. The edge filtering technique removes flat surfaces with poor visual clues, taking advantage of the RGB images captured by the sensor in order to filter the point clouds, working with a representative subset in order to increase the alignment quality.


Promising results were obtained using a freely available dataset. The method was evaluated quantitatively and 
also some images of the 3D reconstructed scenes were obtained for a visual inspection of the results.

\section{Future Work}

  Results can be greatly improved adding an outlier detection method, because one incorrect align between two frames can ruin the complete result. These problems can occur 
when one of the frames exhibit poor visual features or there is an abrupt change in the sensor movement. 

  Another important part of the system that can be improved is the pose graph, where the quality of the restrictions 
determines the quality of the end result. For example, removing some restrictions and adding restrictions obtained with a 
different alignment method or performing some parameter tunning of the proposed method, in 
order to obtain better relative alignment when certain conditions are satisfied.

