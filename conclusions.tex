

This thesis provides a registration system for RGB-D frames that uses geometrical and visual information 
to perform the alignment between related frames.  

A first estimation of the relative transformation between two suscesive frames was obtained using 
visual information, using optical flow and SURF methods. Edge filtering over the RGB image was applied, 
using the visual information to detect a representative subset 
of the data. Combining this with the geometrical information provided by the depth map, filtered point clouds 
where obtained and aligned using the ICP algorithm. 

Using a representative subset instead of the complete point cloud in order to perform the alignment of frames 
reduces necessary calculations and exhibit noticeable improvements in the result. The differences become visible 
when adding constraints between non-consecutive frames, in this case using the filtered point cloud instead of 
the full point cloud allows performing a better alignment, being less prone to get stuck in a local optimum. The 
proposed filtering reduces the amount of points without correspondences between the two point clouds and removes 
flat surfaces that with a bigger amount of points can dominate alignment towards an incorrect result.


Promising results where obtained using a freely available dataset. The method was evaluatated quantitatively and 
also some images of the 3D reconstructed scenes where obtained for a visual inspection of the results.

\section{Future Work}

  Results can be greatly improved adding some outlier detection method and making the corresponding correction when one 
outlier appears, because one incorrect align between two frames can ruin the complete result. These problems can happen 
when one of the frames exhibit poor visual features or there is an abrupt change in the sensor movement. 

  Another important part of the system that can be improved is the pose graph, where the quality of the restrictions 
determines the quality of the end result. For example removing some restrictions and adding restrictions obtained with a 
different alignment method or performing some parameters tunning of the proposed method, in 
order obtain better relative alignment when certain conditions are satisfied.

