
\chapter*{Abstract}

3D point cloud registration is an important step in every 3D reconstruction system. This thesis proposes to register 
3D point clouds, applying a filtering step where only edges that appear 
in the two point clouds to be aligned pass the filter. For this purpose a combination of visual and geometrical information is 
used along with the Iterative Closest Point (ICP)
  algorithm and a pose graph optimization method. The proposed technique reduces the amount of calculations 
involved because only between 10\% and 20\% of original points are used as input for the ICP algorithm, increasing 
the quality of the alignment, working with the most representative subset of the data. At difference to other techniques 
involving edge filtering along with ICP, this proposal increments the odds of a correct alignment filtering out non common 
data from the pair of point clouds to be aligned. Quantitative results shows the advantages of the proposed method. A 
public available dataset was used, 
providing the source 
code and all the necessary information to compare and replicate the presented results.

\section*{Keywords}

3D reconstruction, simultaneous location and mapping, iterative closest point, Kinect, point cloud
