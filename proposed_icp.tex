\section{Proposed Algorithm}
\begin{algorithmic}[1]
\State init R,t using Optical Flow
\State A = sobelFilter(A)
\State B = sobelFilter(B)
\State A' $\leftarrow$ transform(A,R,t) 
\State p $\leftarrow$ closestPoints(A',B)
\State $\{R,t\} \gets$ updateTransformation(p)
\State $e_i = meanSquareError(p)$
\If {$e_i < umbral$ OR  $i > maxIterations$} 
	\State return R,t
\Else
	\State goto step 4
\EndIf
\end{algorithmic}


In the first step the proposed algorithm uses optical flow to obtain a first estimation of 
R,t for the ICP algorithm. The used camera captures both depth and color 
information, for each pair of consecutive color captures optical flow 
is applied, obtaining pairs of correspondences between both captures. 

The domain of optical flow is on the 2D image space, but using 
the 3D information from the depthmap its possible to get the 3D position 
of the each pair of correspondences respect to the camera. 

Having pairs of 3D points, one point of the capture at time t and other point 
of the capture at time t + 1 its possible to obtain the rotation R and translation t
 that minimizes the distances between the correspondences. Obtaining the initial guess 
 for ICP.


Then both point clouds are filtered, removing plain surfaces, thus obtaining point clouds 
with a lesser amount of points. With this ICP will work on point clouds that contains around 
only 3\% of the original points. But this points are highly representative for registration purposes.

Finally the classical ICP algorithm is applied to the point clouds.

 






