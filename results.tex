The dataset \cite{sturm12iros} was used to get a quantitative evaluation 
of the proposed solution. The dataset consists of 39 sequences recorded in
two different indoor environments. Each sequence contains
the color and depth images, as well as the ground truth
trajectory from a motion capture system. The dataset also includes evaluation 
scripts.

The software was programmed in C++ using Point Cloud Library (PCL) and OpenCV.

The Point Cloud Library (PCL) contains an ICP implementation, this implementation 
was modified accord to this proposal.

OpenCV contains most common image processing algorithms and it was used to calculate 
Optical Flow from dataset images. 


The program was executed on a notebook with Ubuntu 13.10 Operating System, 
CPU Intel i5-3210M CPU @ 2.50GHz, 6 GB of RAM 
and a dedicated GPU Nvidia 650M of 2GB. Since software optimization is beyond of the scope 
of this thesis, no optimizations where perfomed and the software just used the CPU to 
made the calculations.



The software has very good results in small image sequences (less than 40 captures) until 
accumulated error grows noticeably. A very basic graph optimization approach was used, in 
order to mitigate the problem.

//small sequence image 


//drift image


Since a very basical approach was used to perform pose graph optimization results still 
exihibit a high amount of accumulated error. However proposed solution exihibit a very 
good performance in small trajectories.
