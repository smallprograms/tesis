
\chapter*{Resumen}

En general un sistema de reconstrucci\'on 3D tiene tres pasos principales: adquisici\'on de datos, registro y reconstrucci\'on. Durante el paso de registro los puntos 3D son registrados en un sistema de coordenadas com\'un. El algoritmo de b\'usqueda iterativa de puntos cercanos (ICP) es 
ampliamente usado para realizar el registro de los puntos, sin embargo es computacionalmente costoso y propenso a converger a un m\'inimo local. Esta tesis propone enfrentar estos problemas, aplicando un paso de filtrado donde solo los bordes que aparecen en las dos nubes de puntos 
que est\'an siendo alineadas pasan el filtro. Una combinaci\'on de informaci\'on visual y geom\'etrica es 
usada junto con ICP y un m\'etodo de optimizaci\'on de grafos. 
La t\'ecnica propuesta 
disminuye la cantidad de c\'omputos necesarios porque s\'olo entre un 10\% y un 20\% de los puntos originales son usados 
como entrada para el algoritmo ICP, 
aumentando la calidad de la alineaci\'on, trabajando con el subconjunto de datos m\'as representativo. A diferencia de otras 
t\'ecnicas que involucran filtrado de bordes junto con ICP, esta propuesta incrementa las probabilidades de una alineaci\'on 
correcta quitando los puntos no comunes entre el par de nubes de puntos a alinear. Los resultados cuantitativos muestran 
las ventajas del m\'etodo propuesto. Un conjunto de datos p\'ublicamente disponible fue utilizado para los experimentos, el c\'odigo 
y toda la informaci\'on necesaria para comparar y replicar los resultados es provista.


\section*{Palabras clave}


Reconstrucci\'on 3D, localizaci\'on y mapeo simult\'aneo, algoritmo de b\'usqueda iterativa de puntos cercanos, Kinect, nube de puntos.
