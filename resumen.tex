
\chapter*{Resumen}

El registro de nubes de puntos 3D es una parte importante en todo sistema de reconstrucci\'on 3D. Esta tesis propone 
registrar nubes de puntos 3D, aplicando un paso de filtrado donde s\'olo los bordes que aparecen en las dos nubes de puntos 
que estan siendo alineadas pasan el filtro. Una combinaci\'on de informaci\'on visual y geom\'etrica es 
usada junto con el algoritmo de b\'usqueda iterativa de puntos cercanos (ICP) y un m\'etodo de optimizaci\'on de grafos. 
La t\'ecnica propuesta 
disminuye la cantidad de c\'omputos necesarios porque s\'olo entre un 10\% y un 20\% de los puntos originales son usados 
como entrada para el algoritmo ICP, 
aumentando la calidad de la alineaci\'on, trabajando con el subconjunto de datos m\'as representativo. A diferencia de otras 
t\'ecnicas que involucran filtrado de bordes junto con ICP, esta propuesta incrementa las probabilidades de una alineaci\'on 
correcta quitando los puntos no comunes entre el par de nubes de puntos a alinear. Los resultados cuantitativos muestran 
las ventajas del m\'etodo propuesto. Un conjunto de datos p\'ublicamente disponible fue utilizado para los experimentos, el c\'odigo 
y toda la informaci\'on necesaria para comparar y replicar los resultados es provista.


\section*{Palabras clave}


Reconstrucci\'on 3D, localizaci\'on y mapeo simult\'aneo, algoritmo de b\'usqueda iterativa de puntos cercanos, Kinect, nube de puntos.
