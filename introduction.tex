\indent The 3D scene reconstruction problem consists of taking the
necessary information from a real scene in order to reconstruct
it in a three dimensional space, usually to be displayed 
in a computer. 

The goal is to represent in the most accurate way the geometric scene details, obtaining rich 
information about the scene that is not explicitly contained in a single image, and then use this 
to reconstruct the scene or to perform a more advanced task that depends on the scene geometry. 
3D scene reconstruction has several applications, for example in the field of autonomous robot navigation it is crucial to have 
the scene geometry in order to locate paths, obstacles and know current robot location, this problem is known as 
Simultaneous Localization and Mapping (SLAM) \cite{SLAM06}. Another application of 3D scene reconstruction is augmented reality, 
where a virtual 3D object is added to a real video of the scene. 3D scene reconstruction is also useful in parts inspection 
in a manufacturing plant, where it is necessary to detect
 fabrication defects on some objects, another application is in statues and buildings preservation, in order to have a digital representation 
of the objects and being able to reproduce or maintain them. There are many areas where 3D scene reconstruction is important \cite{apps2009}.

 
In general the information is acquired
 from the scene with optical devices, such as RGB cameras or depth sensors.
Most of 3D scene reconstruction methods can be classified into passive methods and active methods \cite{lanman}.
Passive methods works without controlling the light in the scene, the sensors used by these methods just receive light (ordinary cameras). 
On the other hand, active methods alter the light in the scene, with a light transmitter and its corresponding 
receptor, projecting patterns of light in order to simplify the matching process prior to the triangulation. 
There are also some active methods that touch the object in order to reconstruct it, but they are beyond 
the scope of this thesis. 

Accurate and expensive equipment to perform 3D scene reconstruction are commercial laser scanners,
 but nowadays there are emerging cheaper devices that 
potentially could perform 
a similar reconstruction at a lower cost \cite{kinectReview13}.

One of the most classical approaches to perform 3D reconstruction is to use multiple 2D
 images taken from similar camera viewpoints and estimate the distance of each
 relevant pixel with triangulation (Stereo Vision, Multiview Vision), with these 
approaches the depth map must be generated using the geometrical information contained
 in the images \cite{Newcombe10livedense}. Nowadays, it is common to find devices that generate depth maps accessible
 to everyday users, automating the depth map generation step, these devices are called depth sensors. 
Depth sensors give to each pixel of the image a depth value, related
with the distance of the real object from the sensor, offering  more
accurate data to perform 3D reconstruction. With the appearance
of cheap depth sensors for gaming and entertainment, there is a growing interest in the development of low cost
3D reconstruction systems. 


\section{Goal of the Thesis} 

One of the key steps of a 3D reconstruction algorithm is the registration, where the 3D points corresponding to
 the scene are  registered in a common coordinate system, preserving the original scene disposition. This step is 
 fundamental in order to being able to reconstruct the scene or recover the sensor trajectory.

The goal of this thesis is to propose a new technique to register point clouds in a common coordinate system, applying 
filtering to the data and combining visual with geometrical information, in order to obtain better results with less 
computational costs. 





