
The 3D scene reconstruction problem consists in take the
necessary information from a real scene in order to reconstruct
it in a three dimensional space, usually to be displayed 
in a computer. 

The objetive is to represent in the most accurate way the geometric scene details, obtaining rich 
information about the scene that is not explicity contained on a single image and then use this 
to reconstruct the scene or use it to perform a more advanced task that depends of the scene geometry. 
Some applications of 3D scene reconstruction are : Autonumous vehicle or robot navigation, where there is crucial to have 
the scene geometry in order to locate paths and obstacles. Augmented Reallity, where a virtual 3D object
 is added to a real video of the scene. Parts inspection in a manufacturing plant, where is necesary to detect
 fabrication defects on some objects. Statues and Buildings preservation, in order to have a digital representation 
of the objects and beign able to reproduce or mantain them.

 
In general the information is acquired
 from the scene with optical devices, such as RGB cameras or depth sensors.
Most of 3D scene reconstruction methods can be classified in passive methods and active methods \cite{lanman}.
The passive methods works without controlling the light in the scene, they just receipt light (ordinary cameras). 
By the other hand active methods alter the light in the scene, with a light emisor and its corresponding 
receptor, projecting patterns of light in order to simplify the matching process prior to the triangulation. 
There are also some active methods that touch the object in order to reconstruct it, but they are beyond 
the scope of this thesis. 

The most accurate and expensive method to perform 3D scene reconstruction is the Laser Scanning, because it has 
an hight accuracy (about xx mm). But nowadays there are emerging cheaper devices that potentially could perform 
a similar reconstruction at a pair of orders of magnitude low cost, due to this there are every year new publications 
related to this low cost reconstruction systems and its known that the main drawback is on the computational power and 
the used algorithms, because using the same devices more accurate reconstructions are performed at each time. 

One of the most classical approaches is to use multiple 2D
 images taken from known camera viewpoints and estimate the distance of each
 relevant pixel with triangulation (Stereo Vision, Multiview Vision), with this 
approaches the depth map must be generated using the geometrical information contained
 on the images. Nowadays is common to find devices that generate depth maps accesible
 to everyday users and automatize the depth map generation step, this devices are called depth sensors. 
Depth sensors give to each pixel of the image a depth value, related
with the distance of the real object from the sensor, offering a more
accurate data to perform 3D reconstruction. With the appearance
of cheap depth sensors for gaming and entertainment (such as
Kinect). There is a growing interest in the develop of low cost
3D reconstruction systems. 


