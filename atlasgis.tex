The ATLAS experiment operates in a multigrid environment, which
consists of three different grid infrastructures: EGEE, OSG and NDGF.
All these sub-grids have different middlewares, therefore different
Grid Information System (GIS) implementations. 

Due to the multigrid nature of ATLAS, a new layer was created -- called 
{\itshape supra-middleware} layer in this thesis-- to provide the coordination and
integration of the three sub-grids that compose the ATLAS grid.
For instance, {\itshape Distributed Data Management} is the component
of this layer that enables the data movement and
{\itshape Monitoring and Accounting} are the components that allows to
know quantitatively how are being used the resources of the whole
ATLAS grid.

Regarding GIS, still there is not a global system capable of
delivering current and reliable information regarding grid topology,
resources, and services of the whole ATLAS grid.
It is required a GIS component in the supra-middleware that 
merges the information coming from different sources and 
ensures
the {\itshape consistency} and {\itshape integrity} of the data, and
all these
dealing in the the highly dynamic, large-scale and multigrid
environment of the ATLAS experiment.


The {\itshape ATLAS Grid Information System} described in this chapter
attempts to overcome this lack, proposing an architecture 
 and implementing a system that collects the
information from diverse sources and provides different tools to be
used by the ATLAS grid components and users.

%The proposed architecture is based on the {\itshape Restful} style,
%which provides simplicity of publishing and consuming a service.

%The GIS of the ATLAS grid provide information on resources and
%services, but they do not provide information related with the topology of
%the ATLAS grid.




%, which implies the need to
%provide a new component in the supra-middleware layer, in order to

%\section{ATLAS Grid Architecture}
%As mentioned above, the ATLAS grid is composed of three different
%sub-grids: EGEE, OSG and NDGF, each providing their own middleware,
%which implies different set of tools to manage distributed data,
%different jobs submission methods, diverse GIS, monitoring and
%accounting systems.

%This scenario implies the need of provide a new layer --called
%supra-middleware-- that coordinates and integrates the three
%middlewares belong to each sub-grid, in order to have a ``a grid of
%grids'', that is, different grid infrastructures working together to
%provide a seamless grid infrastructure. %Therefore, it is necessary to
%have
%%services that manage the data and job submission, as a whole grid.

%The following sections describe the main components that belong to the
%supra-middleware layer:
%\begin{itemize}
%\item Distributed Data Management
%\item Production System
%\item Distributed Analysis
%\item Monitoring and Accounting
%\end{itemize}


%\subsection{Distributed Data Management}
%\label{sec:ddm}

%The \textbf{D}istributed \textbf{D}ata \textbf{M}anagement (DDM) is
%the component in charge to manage the access to ATLAS experiment data
%that is distributed at sites all around the world.
%DDM is the central link between all the components of the ATLAS grid,
%because of data access is needed for any processing and analysis step.

%The DDM system is composed of different components which interact with
%each other to perform the different {\itshape bookkeeping} and
%{\itshape data movement} tasks. 
%The catalogs take care of the first part - bookkeeping - while the
%site services try to answer the different user requests to spread
%existing data among the different sites --called {\itshape
%subscriptions} \cite{monitoringDDM, Lassnig:2007}.

%\begin{description}
%\item[Central Catalogs:] The central catalogs define the {\itshape
%datasets} and their locations. 
%The dataset is an aggregation of data (one or more physical file),
%which are processed together and serve collective as input or output
%of a computation or data acquisition process.  
%A dataset is uniquely identified by a dataset name (dsn) and a
%version, and it expects $O(1000)$ new dataset per day.
%%Last line from DATA CATALOGS, buenos aires workshop

%There are four different catalogs: {\itshape Content, Location,
%Repository} and {\itshape Subscription}. 
%The first one, tracks relationships between files and the coarser
%grained unit of dataset, the Location catalog stores the sites where
%replicas of data can be found and additional information about its
%completeness, the Repository catalog keeps track of existing dataset
%and their different versions, and finally, the Subscription catalog,
%which stores data movement requests from users.

%\item[Site Services:]The site services are collections of agents, and
%their principal goal is to satisfy the subscription. 
%There are agents fetching new subscriptions from the central catalogs,
%evaluating appropriate replicas for each file in a dataset
%subscription, submitting transfers to the File Transfer Service (FTS),
%registering files in the local storage catalogs once they are
%successfully transferred into the site or registering new dataset
%locations in the central catalogs once subscriptions are complete.

%The subscription is the process used to move massive amounts of data
%between sites, on scheduled data movements. 
%Subscriptions are typically used to ship data from data taking from
%the Tier-0 to the Tier-1s and Tier-2s.  
%For instance, if there is a dataset A presented in a site Y but not in
%the site X, and X needs dataset A, it must to subscribe to dataset
%A. 
%Later A is transferred to site X and registered properly in catalogs.
%\end{description}


%The DDM main components are shown in the Fig. \ref{fig:DDMarchitecture}

%\begin{figure}[!htbp]
%\centering
%\pgfuseimage{DDMarchitecture}
%\epsfig{file=images/DDMarchitectur.eps,scale=0.55}
%\caption{DDM Architecture}
%\label{fig:DDMarchitecture}
%\end{figure}


%The datasets have different states in the data flow process: 

%\begin{description}
%\item[Open:] A dataset is in Open state when new files can be added or
%old ones removed.
%\item[Closed:] This state corresponds the latest version of the
%dataset is finished and changes can only be done by reopening it.
%\item[Frozen:] This states occurs when no more changes can
%be done in a dataset.
%\end{description}

%Furthermore, the life-cycle of a dataset subscription can be in
%different states.  
%From the time a new subscription is added in the central catalogs to
%the time the dataset is fully available on the destination site there
%are several states. %, which are shown in the full state machine of
%%the
%%Fig. \ref{fig:machineStateDataset}.

%%\begin{figure}[!htbp]
%%\centering
%% \epsfig{file=images/machineStateDataset.eps,scale=0.55}
%%\caption{Dataset Subscription State Diagram}
%%\label{fig:machineStateDataset}
%%\end{figure}

%\begin{description}
%\item[Queued:] The dataset has been taken for processing by
%the site services, but its contents have not yet been retrieved from
%the content catalog

%\item[Incomplete:] Dataset contents are known, and files have been
%queued locally for transfer where appropriate

%\item[Canceled:] The user canceled the subscription. 
%Files already transferred are not removed, but the ones still in the
%queue will not be replicated

%\item[Broken:] The subscription is invalid and the site services have
%given up on serving it. 
%This is usually due to the wrong definition of the dataset itself

%\item[Complete:] The subscription has been fully served, and all
%dataset files are stored and registered in the destination site
%\end{description}

%DDM  must keep data integrity and provide routine data transfer, data
%monitoring and data access to the ATLAS physics community. 





%\subsection{Production System}
%The ATLAS \textbf{Prod}uction \textbf{Sys}tem (ProdSys) has been
%developed to perform the simulation data production of the experiment
%using grid resources.
%It provides a robust framework to execute a large number of jobs in
%the grid infrastructures.

%The ATLAS experiment requires a large amount of simulated data.  The
%ProdSys component allows to manage the production of large amount of
%Monte Carlo data using grid resources.

%ProdSys is implemented in a modular way in order to used the ATLAS
%grid resources that belong to the three different sub-grids, providing
%an interface layer on top of the various grid middleware used in
%ATLAS.
%It allows to manage the production of a large scale amount of Monte
%Carlo data using grid resources in an automatic way, within minimal
%human intervention \cite{DAFirst}.


%The \textbf{P}roduction \textbf{AN}d \textbf{D}istributed
%\textbf{A}nalysis system (PANDA) is a job management system, and it
%was designed for ProdSys and Distributed Analysis components.  
%It is highly automated, has an integrated monitoring system, and
%requires low operation manpower.
%A description of PANDA is given as follows.


%\subsubsection{PANDA}
%\label{sec:PANDA}
%PANDA has been developed to meet ATLAS requirements for petabyte scale
%production and distributed analysis processing for the ATLAS
%Experiment \cite{pandaPaper}.

%PANDA was designed to support both managed production and user
%analysis via flexible job specifications and injection. 
%The dataset based organization of PANDA matches the DDM system and the
%analysis work model.

%In the PANDA system the jobs are submitted to the PANDA server via a
%simple Python/HTTP client interface. 
%The PANDA server is the main component which provides a task queue
%managing all job information centrally. 
%The PANDA server receives jobs through the client interface into the
%task queue , upon which a brokerage module operates to prioritize and
%assign work on the basis of job type, priority, input data and its
%locality, and available CPU resources. 
%Allocation of job blocks to sites is followed by the dispatch of input
%data to those sites, handled by a data service module interacting with
%DDM. %The
%%implementation of the PANDA server is a LAMP stack with a
%%multi-processing/multi-threading architecture running 50 child-process
%%concurrently. The apache 


%\begin{figure}[!htbp]
%\centering
%\pgfuseimage{pandaOverview}
%\caption{Overview of the PANDA System}
%\label{fig:pandaOverview}
%\end{figure}

%The scheduler sends pilots to batch systems and grid sites. There are
%three kind of scheduler: CondorG scheduler, Local scheduler, and
%Generic scheduler.  
%%The CondorG scheduler sends pilots to most
%%US-ATLAS OSG sites with the CondorG infrastructure. The Local
%%scheduler uses local-batch commands like {\itshape qsub} for PBS and
%%{\itshape condor\_submit} for Condor, which results in very high
%%efficiency and robustness. The Generic scheduler has been developed to
%%support also non-ATLAS OSG VOs and EGEE, and is being extended through
%%OSG extensions project to support Condor-based pilot factory.
%The idea is to move pilot submission from a global submission point to
%a site-local pilot factory.%, whic itself is globally managed as a
%%Condor glide-in.

%Pilots are pre-scheduled to batch system and grid sites.  
%They retrieve jobs from the PANDA server to run the jobs as soon as
%CPU's becomes available. 
%Pilots use resources efficiently; they exit immediately if no job is
%available and the submission rate is regulated according to
%workload. 
%Each pilot executes a job, detects zombie process, reports job status
%to the PANDA server, and recovers failed jobs.

%The PANDA system has a pipeline structure running data-transfer and
%job-execution in parallel.  
%DDM takes care of actual data-transfers: the PANDA server sends
%requests to DDM to dispatch input files to Tier2's or to aggregate
%output files to Tier1's. 
%DDM transfers files and then sends notifications to the PANDA
%server. 
%Jobs wait in the task queue after they are submitted to PANDA, and get
%activated when all input files are transferred. 
%Pilots pick activated jobs and can run immediately since DDM has
%already pre-staged the input files. 
%Similarly, pilots copy output files to local storages and exit
%immediately to release CPU's. 
%Then the PANDA server sends requests to DDM to move files over the
%grid. 
%That is, pilots occupy CPU's only during execution of the ATLAS jobs
%and access to the local storage.

 


%\subsection{Distributed Analysis}
%The \textbf{D}istributed \textbf{A}nalysis (DA) component has as main
%goal to bring the computation power on global scale to individual
%ATLAS physicists by enabling them to easily enter into the vast
%computing resource provided by the different sub-grids for their
%analysis activities, hiding the complexity of the multigrid
%environment.

%The data generated by the ATLAS experiment is distributed in various
%computing facilities (tier structure). 
%The user jobs are in tourn routed depending on the availability of
%relevant data. 
%A typical analysis job consists of a Python script that configures and
%executes a user defined algorithm in Athena\footnote{Athena is an
%ATLAS framework that allows to perform physics analysis. It is an
%oriented based software, primarily using C++ programming language, but
%also with some FORTRAN and Java components.}
%The script specifies input data and produces one or more files
%containing plots and histograms \cite{DAFirst}.

%The DA component has two main frontend clients for the job submission:
%{\itshape Ganga} and {\itshape pAthena}, which are described as
%follows.

%\subsubsection{Ganga}

%Ganga is a frontend tool for job definition and management and mainly
%it enables to the users to submitt jobs to a variety of
%resources. These resources can be the local machine, batch systems
%(PBS, LSF, SGE, or Condor), grid systems (LCG, gLite, NorduGrid) or
%workload management systems (Dirac, Panda).


%Ganga \cite{ganga} is being developed as a grid user interface. It
%allows trivial switching between running test jobs on a local batch
%system and running large-scale analyses on the Grid, hiding Grid
%technicialities; it provides job splitting and merging, and includes
%automated job monitoring and output retrieval. In addition, Ganga has
%a number of specific features:

%\begin{itemize}
%\item It simplifies configuration of applications based on the Athena
%framework
%\item It locates input data using the DQ2 (a DDM tool) data management system
%\item It makes it easy to define and run user-level productions
%\end{itemize}

%Ganga currently provides two user interface clients: a Command Line
%Interface (CLI) and a Graphical User Interface (GUI). In addition, it
%can also be embedded in scripts for non-interactive/repetitive use.

%A job in Ganga is constructed from a set of building blocks. %(see
%%Fig. \ref{fig:gangaJob}). 
%All jobs must specify the software to be run (application) and the
%processing system (backend) to be used.  
%Ganga provides a framework for handling different types of
%application, backend, dataset, splitter and merger, implemented as
%plugin classes. 
%Each of these has its own schema, which places in evidence the
%configurable properties.

%%\begin{figure}[!htbp]
%%\centering
%% \epsfig{file=images/gangaJob.eps,scale=0.55}
%\caption{A job in Ganga}
%%\label{fig:gangaJob}
%%\end{figure}

%\subsubsection{pAthena}
%pAthena is a python script designed to enable access to OSG resources
%via PANDA job management system \cite{DAFirst}.

%pAthena is an interface to the ATLAS offline software framework,
%Athena, and it makes the submission of analysis jobs to the PANDA
%system in two steps:
%\begin{itemize} 
%\item {\itshape Build step:} It gathers up user code, stores it and
%ships it to processing site for (http) retrieval by the pilot and site
%installation/build in user area.
%\item {\itshape Execution step:} It runs N Athena jobs with
%user-designated input and output datasets.
%\end{itemize}



%\subsection{Monitoring and Accounting}

%\textbf{Monitoring} is the component that provides a uniform and
%complete view of various activities like job processing, data movement
%and publishing, access to distributed databases regardless of the
%underlying multigrid environment (EGEE, OSG and NDGF)
%\cite{monitoringDash}.

%%allows to know
%%quantitatively how are being used the grid resources, which helps to
%%understand the performance limitations and helps to evaluate future
%%development work.

%ATLAS Monitoring provides monitoring information in a transparent way
%and combines data related to the performance of the Grid
%infrastructure and data that is activity or application specific.
%Such a monitoring system is able to satisfy users in various roles
%including production manager, coordinator of data transfer, site
%administrator supporting a given experiment at the local site or
%simply an analysis user submitting analysis jobs to the grid.

%ATLAS Monitoring covers a wide range of activities:

%\begin{description}

%\item[Job Monitoring:] 
%Its goal is to keep track of all jobs submitted by the experiment
%users and stores the main monitoring indicators, such as resource
%usage and sharing, grid behavior application robustness and data
%access quality.

%\item[Site Reability:] 
%This activity facilitates the task of the site administrators, to
%estimate the site performance regarding job processing and data
%transfer and to detect and understand eventual problems at the site.

%\item[Task Monitoring:] 
%It is as a supplement of the job monitoring application but focused on
%the needs of the LHC analysis users submitting their jobs to the grid.
%It provides a consistent way of following a user's analysis tasks
%regardless of the job submission tool, which was used for the job
%generation and submission.

%\item[DDM Monitoring:] 
%DDM has different components interacting to perform the data
%bookkeeping and placement task. DDM Monitoring is a central service
%which collects all the callbacks sent by the different agents in the
%DDM site services. It provides monitoring information, summaries and
%statistical data, which are exposed to the users.

%\item[ATLAS Production Monitoring:] 
%It holds information regarding job status, such as waiting submission,
%scheduled in a resource, running, being aborted, among others.
%\end{description}


%Thus, ATLAS Monitoring allows to follow activities like job processing,
%data transfer, data access and access to distributed data bases. 
%It keeps track of all jobs submitted by the users and stores the main
%monitoring indicators such as resource usage and sharing, grid
%behavior, application robutness and data distribution.  
%The information related to the job processing can be aggregated and
%presented per user, per site or computing element, per resource
%broker, per application or per input collection.

%%The monitoring component shows relevant quantities of the current
%%state of the experiment activities on the grid, for example, how many
%%jobs are running, pending and accomplished, and which fraction of the
%%accomplished jobs were successful. Main quantities in resource
%%utilization are CPU, memory consumption and input-output rates.



%%In the ATLAS grid environment, jobs submitted by the users may be sent
%%to computing resources close to the data or may go to remote resources
%%with available job slots, thus reducing queue times.  
%%As a consequence, jobs that run on a grid environment must be properly
%%accounted for. 
%%Here, the \textbf{Accounting} componentis fundamental to be aware
%%about the usage of resources (CPU time, wall-clock time and memory).

%The ATLAS \textbf{Accounting} component also attempts to provide a
%global view, working one level above the accounting systems of each
%sub-grid and providing a view considering the ATLAS grid as a whole
%(not per separated sub-grid) and also providing the information in
%different formats and specific views (considering the ATLAS topology).
%The Accounting component is one of the contribution of this thesis and
%it is discussed in detail in Chapter \ref{accsys}.


%The Monitoring and the Accounting components seem to be the same, but
%they are different.  
%The monitoring components plays a role in {\itshape realtime}, in
%other words, there are some services retrieving the data and exposing
%this data as soon the jobs are being sent to the grid, and by the
%other hand, the accounting component retrieves the data time after the
%job was sent (for instance using the logs files), because the idea of
%the accounting component is to perform subsequently analysis of the
%usage of the grid resources, in order to elaborates more sophisticated
%and complete reports to know in deep the grid behavior.\\



%%%%%%%%%%%%%%%%%%%%%%%%%%%%%%%%%%%%%%%%%%%%%%%%%%%%%%%%%%%%%%%%%%%%%
%\begin{com}
%SEVERAL SUBSECTIONS TO BE COMPLETED!!!
%\end{com}
%%%%%%%%%%%%%%%%%%%%%%%%%%%%%%%%%%%%%%%%%%%%%%%%%%%%%%%%%%%%%%%%%%%%%


%%%%%%%%%%%%%%%%%%%%%%%%%%%%%%%%%%%%%%%%%%%%%%%%%%%%%%%%%%%%%%%%%%%%%
%\begin{com}
%12.05.2008 REVISADO HASTA AQUI. YA ES TARDE,
%ESTOY CANSADO Y TENGO SUE. NECESITO UNAS MINI-VACACIONES,
%PERO NO DIVISO NINGUNA VENTANA DE DESCANSO EN EL HORIZONTE.
%\end{com}
%%%%%%%%%%%%%%%%%%%%%%%%%%%%%%%%%%%%%%%%%%%%%%%%%%%%%%%%%%%%%%%%%%%%%

%DDM, ProdSys, DA, Monitoring and Accounting are components that belong
%to the supra-middlare layer, whose purpose is to provide the
%coordination and integration of the three sub-grids that compose the
%ATLAS grid.  
%The ATLAS experiment has called \textbf{{\itshape ATLAS Distributed
%Computing}} the area in charge of manage all the components previously
%described.

%However, it is still a need of a global component that provides services
%related with the delivery of infomration regarding resources, services
%and topology of the ATLAS grid.
%The following section describes the current services that attempts to
%provide that information.

\section{Grid Information Services in ATLAS}
\label{sec:sources}
Currently, the ATLAS Distributed Computing does not have a global
component that provides the services related with the collecting and
delivery of data related with resources, services and topology of the
ATLAS Grid. 
However, the ATLAS grid requieres these services to perform a correct
and efficient work, therefore, ATLAS relies on several services in
order to gather information about grid resources:

\begin{itemize}
\item Berkeley Database Information Index (BDII)
\item Grid Operation Center Database (GOCDB)
\item NorduGrid Information System
\item OSG Information Management (OIM), 
\item PANDA Database
\item TiersofATLAS (ToA)
\end{itemize}

Some of these services provide information about grid resources,
others provide information about grid services, and some of them
provide information about topology, but the latter is not provided in
a automatic and efficient way.

The items mentioned above are described as follows.

%In the following sections we describe each of them.

\begin{description}
\item[TiersOfATLAS:]
As its name says, {\itshape TiersOfATLAS} (ToA) provides information
regarding the tiers of the ATLAS grid, including a mapping between
{\itshape DDM sites} and the tier and cloud which they belong (topology
information, only for Tier-1 and Tiers-2).
Furthermore, it provides information regarding Storage Element services
of the ATLAS sites.

A site that provides the Storage Element service implies that 
provides storage capacity and the services that allows uniform access
to the data storage resources. 
These storage resources are split into virtual areas, which are called
{\itshape space tokens} (also called {\itshape DDM sites}).

For instance, the CERN-PROD site corresponds to the biggest resource
center of the ATLAS grid and is part of the Tier-0, and provides the
Storage Element service, which has six space tokens: {\itshape
DATADISK, DATATAPE, MCDISK, MCTAPE, SPECIALDISK, USERDISK} and
{\itshape USERTAPE}.  
This information is provided by ToA.

%It has been described the functionalities of ToA, but 
It is necessary to mention the disadvantages that presents ToA due to
it explains the need of a global and efficient component that
provides information regarding resources, services and ATLAS
topology.

The topology information provided by ToA is not totally complete,
because the mapping is done between the space token (DDM site)
$\rightarrow$ tier $\rightarrow$ cloud, but the mapping should be
space token $\rightarrow$ service endpoint $\rightarrow$ ATLAS site
$\rightarrow$ tier $\rightarrow$ cloud. 
Currently this information is missing and there is not any system that
provides it.

Another disadvantage is that TiersOfATLAS is a simple configuration
Python file, which is manually edited and is copied locally in the
different grid components. 
Thus, each grid component has a local copy of ToA (called
TiersOfATLASCache), which is obtained using the \textbf{C}oncurrent
\textbf{V}ersioning \textbf{S}ystem (CVS)\footnote{The Concurrent
Versioning System is a version control system that keeps
track of all work and all changes in a set of files, and it allows to
several developers work in a same software project. \url{http://ximbiot.com/cvs/wiki/}}. %(see Figure
%\ref{fig:ToACVS}).

CVS is not the best tool to be used in a dynamic grid enviroment as
part of a Grid Information Service, because this tool does not provide
services or {\itshape agents} that allow to have always the most
updated data.
Furthermore, the clients (in this case the ATLAS components) are
responsible to update the data and download the TiersOfATLASCache file
using the command lines tools that CVS provides, therefore, this
information is managed in a {\itshape manually} way --not very proper
for this multigrid environment.

%\begin{figure}[!htbp]
%\centering
%\pgfuseimage{ToACVS}
%\caption{ToA and CVS Overview. ESTA IMAGEN SERA MODIFICADA}
%\label{fig:ToACVS}
%\end{figure}


Thus, ToA does not reflect the changes of the grid resources in an
automatic and periodic manner, producing {\itshape data inconsitency}.
ToA can be seen as a component that belongs to the supramiddleware
layer but does not provide all the necesary GIS services.

%TiersOfATLAS contains information regarding ATLAS topology and the
%Storage Element service, which is mainly use by DDM component.



%The ATLAS grid has around 200 sites, which provides different
%services, among them the {Storage Element} service, which provides
%uniform access to data storage resources (see \ref{gLite}).


%It provides information regarding  the Storage Element services of the ATLAS
%sites that belong to Tier-1 and Tier-2. {\itshape TiersOfATLAS}

%It is a python file describing the topology and the site data in
%python dictionaries.


%%%%%%%%%%%%%%%%%%%%%%%%%%%%%%%%%%%%%%%%%%%%%%%%%%%%%%%%%%%%%%%%%%%%%
%\begin{com}
%SEEMS TO BE TOO SHORT AS A SUBSECTION!!!  
%IT SHOULD BE COMPLETED!!!
%\end{com}
%%%%%%%%%%%%%%%%%%%%%%%%%%%%%%%%%%%%%%%%%%%%%%%%%%%%%%%%%%%%%%%%%%%%%



\item[BDII]

The \textbf{B}erkeley \textbf{D}atabase \textbf{I}nformation
\textbf{I}ndex (BDII) is a Grid Information System, which was 
developed within the LHC Computing Grid (LHC) project as replacement
of MDS2. 
The detailed description of BDII is in Section \ref{secBDII}.

BDII provides information regarding resources and services of the EGEE
and OSG sites. The NDGF sub-grid is particular situation, because in
the ATLAS grid, NDGF is seen as one federated site, {\itshape NDGF-T1},
therefore, the NDGF sites that compose the NDGF-T1 are missing in BDII. 
Furthermore, BDII does not provide information about ATLAS topology.

BDII can be see as a component of the supramiddleware layer, but does
not provide information regarding topology or NDGF sites, thus, it
does not provide all the necesary GIS services.



%\begin{com}
%COMO DESCRIPCIONES GENERALES ESTA BIEN.
%PERO ME PARECE QUE HABRI QUE PROFUNDIZAR MAS INGENIERILMENTE EN
%ALGUNOS ASPECTOS PERTIENENTES Y NECESARIOS PARA TU TESIS.
%\end{com}
%%%%%%%%%%%%%%%%%%%%%%%%%%%%%%%%%%%%%%%%%%%%%%%%%%%%%%%%%%%%%%%%%%%



\item[PANDA Database]
The (\textbf{P}roduction \textbf{AN}d \textbf{D}istributed
\textbf{A}nalysis) (PANDA) is a software system that has been developed
by US ATLAS to meet ATLAS requirements for full scale production and
distributed analysis processing. 
The detailed explanation of PANDA is in Section \ref{sec:PANDA}.
%provides an integrated
%service architecture with 
%{\bf\itshape late (what do you mean here by ``late''?)\/}
%binding of jobs, maximal automation
%through layered services, tight binding with ATLAS distributed data
%management (DDM) system, advanced error discovery and recovery
%procedures, and other features.

%PANDA has been developed by US ATLAS to meet ATLAS requirements for
%full scale production and distributed analysis processing
%\cite{pandaURL}. 

%The clients submit jobs to PANDA server, where the communication
%protocol is independent from grid-middleware, i.e., jobs can be
%sumbmitted from any grid flavor (OSG, EGEE, NorduGrid)
%\cite{pandaTadashi}. 

PANDA has a database, wich contains a table called {\itshape
schedconfig}. This table gets its data from different sources:

\begin{itemize}
\item BDII: 
site information, releases, GOC name (name defined by the Grid
Operation Center).
\item TiersOfATLAS: 
DDM site, SRM endpoint for Storage Element host and path.
\item pilotController.py: 
It corresponds to a component of PANDA that provides some information
about the structure of a cloud.
\end{itemize}

%The mapping between space tokens and sites is done manually in the pilotControler

{\itshape Schedconfig} table is the main source of information
regarding {\itshape Computing Element} (CE) services and {\itshape
PANDA Queues} of the ATLAS sites.
However, this table also contains information regarding Storage
Elements services due to the CE service is responsible for accepting
jobs and dispatching them for execution in the nodes, thus it is
necessary for the CE services to be aware about data location when the
jobs are running.

The space token to GOC site name mapping is currently done manually in
pilotController.py, which certainly is not very efficient, specially
when the ATLAS experiment reaches a state of maturity high enough for
real data taking.




\item[OIM] 
The \textbf{O}SG \textbf{I}nformation \textbf{M}anager (OIM) is a
system that provides information of OSG sites, mainly regarding sites
and their services.

This sytem was developed by OSG, thus does not belong to the
supra-middleware layer, which implies that only supply information of
OSG sites.
To provide some interoperability, OIM provides two files in a CSV
style, which provides some information regarding WLCG sites and their
GOC name and services.

\item[GOCDB:]
The \textbf{G}rid \textbf{O}peration \textbf{C}enter
\textbf{D}ata\textbf{B}ase is the entity responsible for coordinating
the overall operation of the grid. It acts as a central point of
operational information such as configuration information and contact
details. GOCDB has been defined --by the ATLAS management board-- as
the official source of information of the ALTAS site names, but
currently, only EGEE sites are registered in this system.

%GOCDB provides information such as 


\item[ARC Information System]:
It corresponds to the information system of the NDGF sub-grid,
therefore only provides information regarding services and properties
of NDGF sites.


%\subsection{WLCG SAM Framework}

%\textbf{S}ervice \textbf{A}valilability \textbf{M}onitoring (SAM) is a
%monitoring tool conceived as a STF\footnote{See \ref{secSFT}.} (Site
%Functionality Test) extension.
%It has been written in Python (submission framework, logging, error,
%reporting, configuration files), re-using most of the SFT code, with
%the only exception of the web service API framework.
%The heart of the testing and publishing system is an Oracle
%database, which is connected to the SAM sensors (both those
%integrated into the framework and the standalone ones) through the
%Tomcat\footnote{See \ref{secTomcat}.} web services. 
%These services include (GOC DB) query and publishing web services
%implemented in Java or using servlets.  
%In addition to that, the Oracle DB interacts with the top level BDII
%using a Python script \cite{SAMMAnual}.

%\begin{figure}[!htbp]
%\centering
% \epsfig{file=images/SAM.eps,scale=0.55}
%\caption{SAM architecture.}
%\label{figSAM}
%\end{figure}

%The monitoring procedures are carried out using sensors which
%regularly publishe the results for all monitored service nodes and
%that are integrated within the SAM framework mainly following two
%different approaches: 
%\begin{itemize}
%\item Creating a standalone sensor (daemon or cron job) which tests
%  all essential service nodes and publishes the results to SAM using
%  publishing scripts or webservice APIs.
%\item Integrating test script (sensor) with the SAM framework.
%\end{itemize}


%%%%%%%%%%%%%%%%%%%%%%%%%%%%%%%%%%%%%%%%%%%%%%%%%%%%%%%%%%%%%%%%%%%
%\begin{com}
%DESCRPTION TOO SHORT?
%\end{com}
%%%%%%%%%%%%%%%%%%%%%%%%%%%%%%%%%%%%%%%%%%%%%%%%%%%%%%%%%%%%%%%%%%%

\end{description}


\section{Description of the Problem}

%The previous chapter described the ATLAS Grid Architecture and untill
%this point this chapter described the current services that provides
%some functionalities of a Grid Information System.

% in order to have an
%overview of the {\itshape complexity} of the ATLAS grid.  
%The complexity of ATLAS grid is due to the {\itshape multigrid} environment.  
%The ATLAS grid not only have one grid infrastructure, but three, which
%implies a highly dynamic and heterogeneous grid system and diverse
%implementations of the basic services of the grid.
%A grid infrastructure is a model of distributed computing that allow
%the {\itshape large-scale} sharing of storage and computation
%capacity.  
%A grid system is already a complex system due to its heterogeneous and
%dynamic nature. 

The  ATLAS multigrid environment implies different middlewares, therefore,
different information systems and diverse sources of information
providing {\itshape ``GIS data''}. 
A vital aspect of a grid is the information system, %thus, in this
%scenario the {\itshape interoperability} of the systems presents
%problems.
which
%A Grid Information System 
is mainly used to find available resources and services, furthermore,
in the ATLAS grid it is fundamental to retrieve information about its
topology. Currently there is not a proper system that can satisfy
those functionalities, therefore in this thesis it is proposed a new component
called \textbf{A}TLAS \textbf{G}rid \textbf{I}nformation
\textbf{S}ystem (AGIS) aiming to solve those needs.
%The ATLAS grid infrastructure has three different grid infrastructure,
%wich adds extra complexity, in the coordination and integration of
%three different middlewares each
%one with its own implementation of basic services, in particular the
%Grid Information System.

%\item 
%The differences between the grid information systems of the three whole ATLAS grid is based mainly in the 
%\item 
%ATLAS define a hierarchical model A new layer that provides a global
%interface has been added in the ATLAS architecture, which is called
%{\itshape supra-middleware}.  This layer adds new components to the
%system, increasing the



To have an efficient work in the ATLAS grid, % --specially when ATLAS
%reaches a state of maturity high enough for real data taking, various
various ATLAS components need to be aware
%\sout{about} 
of the resources, services and topology of the ATLAS Grid.  
Currently there are some sources of information, but the problem is
that sometimes the information is not complete,
%\sout{the information}
or it is not coherent
%\sout{between them}, 
%\sout{or}
%\sout{and}
the topology information is not present in a complete and efficient
manner.  
Hence it is neccesary to have a single system that allows to retrieve
this information.

AGIS is the proposed model, which attempts to solve the lack of a
single system with information of the topology, services associated,
site availavility, and so on.  
The idea is to maintain the static (and semi-static) information in
order to be used by the various components of the ATLAS Grid
enviroment.  
With regards to the dynamic information (number of jobs, CPU time,
storage capacity used, among others ), Dashboard
Monitoring\cite{monitoringDash} is the system in charge
of this.


%%%%%%%%%%%%%%%%%%%%%%%%%%%%%%%%%%%%%%%%%%%%%%%%%%%%%%%%%%%%%%%%%%%
%\begin{com}
%AQUI ESTAS APUNTANDO AL CORAZON DE TU TESIS.
%�QUIEN PROPUSO ``AGIS''? TU?
%QUE SIGNIFICA AGIS?
%EN LA VERSION FIANAL NO DEBE HABER PUNTOS SUSPENSIVOS
%SON APROPIADOS EN POEMAS O PROSA (llaman al lector a echar a volar
%la imaginacion, las emociones, la sensibilidad, ... (!), en pos del
%vuelo espiritual del poeta), PERO EN POCOS O NINGUN CASO SON
%APROPIADOS DESDE EL PUNTO DE VISTA TECNICO.
%HE AHI UNA DIFERENCIA (MAS) ENTRE EL/LA POETA Y EL/LA TECNICO/A!
%\end{com}
%%%%%%%%%%%%%%%%%%%%%%%%%%%%%%%%%%%%%%%%%%%%%%%%%%%%%%%%%%%%%%%%%%%


\section{Motivational Scenarios}
This section describes some scenarios that shows the need and
usefulness of an information system for the ATLAS grid.

The first scenario shows how is represented currently the ATLAS
topology and how is retrieved this data for the others ATLAS grid
components.

The second scenario shows that the ATLAS Grid Information System can
be used by the Accounting System to retrieve data related with the
topology of the ATLAS grid, wich allows to visualize the accounting
data in the proper way.

The third scenario describes how the Data Distributed Management (DDM)
needs to retrieve information related with the services of the ATLAS
grid.

And finally the last scenario shows why the Panda component needs to
obtain information related with the resources belonging to the ATLAS
grid.



\subsection{ATLAS Grid Topology Representation}
The ATLAS grid structure is complex, it is composed of around two
hundreds of sites that belongs to different grid infrastructures (the
sub-grids), there is a {\itshape multi-tier} hierarchical model that
allows to optimize the use of the resources and furthermore, the ATLAS
sites are divided in clouds.

As was defined in Section \ref{sec:tierStructure}, the ATLAS grid
{\itshape topology}\footnote{{\itshape Grid network topology} concept
referes to the computer network of the grid, the virtual and physical
interconnection of the computers through a particular network . 
This concept is not in the scope of this thesis.}  concept refers to
the sub-grids (EGEE, OSG and NDGF), the tiers, the clouds, and the
sites that belong to the grid, along with the associations between all
these entities.

Currently, some ATLAS grid topology information is present in the
TiersOfATLAS (ToA) file. From this source it is possible to obtain the
mapping between tiers, clouds and {\itshape space tokens}.
Space token is a concept mainly associated to DDM, which
corresponds storage areas provided by the different ATLAS sites.


\begin{figure}
\begin{Verbatim}[commandchars=\\\�\?,frame=single,fontsize=\relsize{-1}]
topology = {
    'ALL': [ 'CERN', 'ITALYSITES', 'SPAINSITES', 'UKSITES', 'FZKSITES', 
        'NLSITES', 'TAIWANSITES','FRANCESITES', 'CANADASITES', 
	'USASITES', 'NDGF' ], (1)
    'TIER1S': [ 'RAL', 'PIC', 'FZK', 'SARA', 'CNAF', 'ASGC', 'LYON', 
        'TRIUMF', 'BNL', 'NDGFT1' ], (2)

    'CERN': [ 'CERN-PROD_DATADISK', 'CERN-PROD_DATATAPE', 'CERN-PROD_DAQ', 
        'CERN-PROD_TZERO','CERN-PROD_MCDISK', 'CERN-PROD_MCTAPE', 
	'CERN-PROD_USERTAPE', 'CERN-PROD_SPECIALDISK' ], (3)
    
    'ITALYSITES': [ 'CNAF' , 'ITTIER2S' ], (4)
    'CNAF': ['CNAFDISK', 'CNAFTAPE', 'INFN-T1_MCDISK', 'INFN-T1_DATADISK', 
        'INFN-T1_MCTAPE', 'INFN-T1_DATATAPE'], (5)
    'ITTIER2S': [ 'MILANO', 'INFN-MILANO_DATADISK', 'INFN-MILANO_MCDISK', 
        'INFN-MILANO_PRODDISK','ROMA1', 'INFN-ROMA1_DATADISK', 
	'INFN-ROMA1_MCDISK', 'INFN-ROMA1_CALIBDISK', 'INFN-ROMA1_PRODDISK',
        'NAPOLI', 'INFN-NAPOLI-ATLAS_DATADISK', 'INFN-NAPOLI-ATLAS_MCDISK',
	'INFN-NAPOLI-ATLAS_CALIBDISK', 'INFN-NAPOLI-ATLAS_PRODDISK',
        'LNF', 'INFN-FRASCATI_DATADISK', 'INFN-FRASCATI_MCDISK', 
	'INFN-FRASCATI_PRODDISK' ], (6)
    ...
\end{Verbatim}
\caption{TiersOfATLAS topology.}
\label{fig:ToACode}
\end{figure}



Figure \ref{fig:ToACode} shows an extract of the ToA code. The
ATLAS grid topology is represented using Python
dictionaries. The {\itshape ALL} (1) key, represents the clouds of the
ATLAS grid --eleven clouds in total. Later the {\itshape TIER1S} (2)
key represents the names of the Tier-1s defined by DDM. 
In (3), the CERN key defines the space tokens that
belong to the CERN cloud. The {\itshape ITALYSITES} (4) key represents
the names of the Tier-1 and Tier-2 of Italy, and in (5) {\itshape
CNAF} represents all the space tokens that belong to the Tier-1 of
Italy. Finally, {\itshape ITTIER2S} (6) represents all the space
tokens that belong to the Tier-2 of Italy. Thus, we can see a
representation of the topology of the ATLAS grid, where it is
considered the clouds, tiers and space tokens, but  {\itshape site} entity representation is missing.\\


In this thesis, it has been proposed a new data model that allows to
represent the relation $cloud \rightarrow tier \rightarrow
site$ (data model is shown in Figure \ref{fig:dataModel}). %Fig. \ref{fig:agisSiteDataModel} shows a part of the ATLAS
%where it is represented the the site entity.
%The ATLAS Grid Information System attempts to solve the aforementioned
%problem. Fig. \ref{fig:agisSiteDataModel} shows a part of the ATLAS
%Grid Information data model, where it is represented the the site
%entity.
%\begin{figure}[!htbp]
%\centering
% \epsfig{file=images/siteDataModel.eps,scale=0.55}
%\caption{Site entity of the ATLAS topology.}
%\label{fig:agisSiteDataModel}
%\end{figure}
For the components of the ATLAS grid it is necessary to know the
relation {\itshape cloud$\rightarrow$tier$\rightarrow$site}. For
instance, the Monitoring System needs to obtain that relation to
publish monitoring data. Using the new {\itshape AGISQuery} API
provided by the ATLAS Grid Information System, the Monitoring System
component can retrieve the complete and correct topology data, and in
the proper format.


\subsection{ATLAS Grid Information System and Accounting System}
%The ATLAS Accounting System allows to retrieve accounting data from
%different sources �from the different sub-grids� providing a global
%view of the accounting data of the ATLAS grid in different formats and
%specific views --considering the ATLAS topology.

The ATLAS Accounting System (Chapter \ref{accsys}) allows to retrieve
accounting data from different sources coming from the different sub-grids,
and exposes the information to the users in different formats,
providing a "topological view", and also allows to add new sources of
information in an easy way due to the modular implementation of the
system.

%The ATLAS Accounting System provides a global view of the accounting
%data of the ATLAS grid and also provides the information in different
%formats and specific views --considering the ATLAS topology. 

The lack of a single information system implies that it is not
possible to display the accounting information in the formats
required, mainly in the topological visualization. 
Currently the ATLAS Accounting System have to retrieve topological
data from different sources (BDII and TiersofATLAS) to give the
topological view to the users, but the goal of this system is to
retrieve accounting data --not topological data-- for wich an ATLAS
Grid Information System is fundamental for the proper work of this
system.


% find out how the resources provided for all the sites
%belonging to the ATLAS grid are being used. This situation should not
%occur in a grid system, in particular, int the ATLAS grid, which is a
%big collaboration composed by diverses countries, each one providing
%different resources. It is necessary to have quantitative mesaurements
%which help to understand the performance limitations and help to
%evaluate future development work.

%To display the accounting data in a topological view, it is necessary
%to count with a system that provides that information. As was mention
%before, TiersOfATLAS is the current source of information that provide
%data of the ATLAS grid topology, but there is a lack in the site
%concept. 


%Thus, we can see again the need of a single system that
%provide this kind of information.





\subsection{Data Distributed Management and Grid Services}
The Data Distributed Management (DDM, Section \ref{sec:ddm}) is a
component of the ATLAS grid which is in charge of the different
bookkeeping and data movement tasks. 
DDM is composed by different components. One of these components is
{\itshape Site Services}, which is a collection of
agents performing different tasks such as fetch new subscriptions from
the central catalogs, evaluate appropriate replicas for each file in a
dataset subscription or submit transfers to the File Transfer Service
(FTS). In particular, in the latest task, DDM needs to know about the
FTS provided by the different sites of the ATLAS grid. Currently, DDM
retrieves service information from TiersOfATLAS, but this source is an
static file, which does not change automatically, which definitely is
not efficient in a distributed environment, which is changing
constantly.

The ATLAS Grid Information System will allow to DDM to query about
FTS and also to retrieve information about all the services of the
ATLAS grid. DDM can obtain this information using the python APIs
provided by the system.

\subsection{Panda and ATLAS Grid Resources}
The Productions and Distributed Analysis System (PANDA, Section
\ref{sec:PANDA}) is a distributed software , which provides an
integrated service architecture with late binding of jobs, maximal
automation through layered services, tight binding with ATLAS
Distributed Data Management (DDM) system, advanced error discovery and
recovery procedures, and other features.

When the users of the ATLAS grid submit a job using PANDA, the PANDA
server receives the jobs and manages all the jobs information
centrally.
The PANDA server must prioritize and allocate work on the basis of the
job type, priority, input data and its locality, and available CPU
resources. The allocation of job blocks to sites is followed by the
dispatch of input data to those sites.

As was described before, PANDA needs to be aware about information
related with the grid resources in order to allocate the jobs in the
proper sites. 
PANDA not only needs resource information, but also need service and
topology information of the ATLAS grid. 
To obtain this information, PANDA must collect the data from different
sources, such as {\itshape schedconfig} table and TiersOfATLAS to
obtain topology and resource data. 
The problem here, is that this sources are not automatized systems, if
there is a change in some resource of the ATLAS grid, this change is
not propagate automatically, which causes that the ATLAS grid does not
work efficiently.\\


%We have described four scenarios where it is possible to see the
%importance of to have an ATLAS Information System that provides
%information related with the resources, services and topology of the
%ATLAS grid. 

We have described four scenarios where it is possible to see the
problems generated when there is not a single information
system. Without a properly working information system it is not
possible to construct a production quality grid system.

The following sections describe the architecture and all
the development stages of AGIS.


\section{AGIS Architecture}
\label{sec:agisArch}
The main goal of the ATLAS Grid Information System (AGIS) is to
provide the missing information regarding resources, services and
topology of the whole ATLAS grid, in order to provide interoperability
in a multigrid environment. AGIS
%interoperability regarding GIS, for which AGIS
retrieves static and semi-static information %regarding resources,
%services and topology of the ATLAS grid,
from the diverse sources coming from the different sub-grids.
%, in order to overcome the
%the diferences between the unde, 
%interoperability in the multigrid environment, exposing AGIS data
%to the different components and users of the ATLAS grid.

AGIS  was designed
considering the benefits of the  Dashboard Framework
\cite{monitoringDash} which provides directives to model a system
with a layered client-server achitecture, following a {\itshape Model
View Controller} \cite{bib:MVC} pattern. 
Furthermore, AGIS follows the \textbf{RE}presentational \textbf{S}tate
\textbf{T}ransfer (REST) principles \cite{fielding}, which are a set
of architectural constraints on top of the basic client-server
architecture style.

%it was added the {\itshape RESTful} style in the design
%of AGIS architecture.

AGIS --the server--  %is a layered client-server architecture because there is server (AGIS itself)
%component that has diverse software components which are provided through a network-accessible endpoint, 
provides diverse {\itshape services} to the clients for data
retrieval. These services are % which corresponds
%On the conceptual level a service is a 
software components provided
through a network-accessible endpoint.

The AGIS architecture (Figure \ref{fig:AGISArchitecture}) groups the
components in three main layers: {\itshape Data Access Layer},
{\itshape Collectors}, and {\itshape Client Connection}, which are in
charge
%, for which presents three main
%layers in charge 
of data management, data retrieval and data
exposure, respectively. 
%The client component corresponds the diverses ATLAS components
%requesting data regarding on site, services and topology of ATLAS
%grid, for wich send a request to the server.
%The server either rejects or performs the request and sends a response
%back to the client.

\begin{figure}[!htbp]
\centering
\pgfuseimage{AGISArchitecture}
%\epsfig{file=images/AGISArchitecture.eps,scale=0.55}
%\pgfuseimage{agisArch}
\caption{AGIS Architecture Overview}
\label{fig:AGISArchitecture}
\end{figure}

The MVC pattern isolates business logic from user interface, resulting in an
application easy to modify either the visualization of the application
or the underlying layers.

REST is considered to be the guiding principles behind the {\itshape
Ultra Large Scale} (ULS) systems due to its principles enable the
achievement of important attributes such as loose-coupling,
reliability, data visibility and interoperability \cite{xu}. 
Thus, REST principles were considered in the design of AGIS
architecture, and they are described as follows.



%The REST principles include 


\subsection{Representational State Transfer}
%An architectural style is a coordinated set of architectural
%constraints that restricts the roles and features of architectural
%elements, and the allowed relationships among those elements,
%within any architecture that conforms to the style. Thus, a style provides a
%name by which we can refer to a packaged set of architectural design decisions
%and the set of architectural properties that are induced by applying the style.

\textbf{RE}presentational \textbf{S}tate \textbf{T}ransfer (REST) was
originally introduced in 2000 in the doctoral dissertation of Roy
Fielding \cite{fielding} and it refers to an architectural style for
building large-scale hypermedia systems.
REST principles works on top of the basic client-server architecutre, they include {\itshape stateless} and {\itshape context-free}
requests, standardized and unified interfaces, and URL identifiable
and inter-linked resources.
In REST every piece of interesting information is exposed as an
{\itshape abstract resource} which is identifiable through a URL and
may have multiple {\itshape representations}.  These resources can be
accessed via a uniform interface (standarized HTTP verbs) and exchange
the representations \cite{xu}.

The main REST principles may be summarized as follows \cite{pautasso}:

\begin{enumerate}
\item \textbf{Resource identification through URL:}
A RESTful Web service exposes a set of resources which identify the
targets of the interactions with its clients. 
Resources are identified by URLs (\textbf{U}niform \textbf{R}esource
\textbf{L}ocators), which provide a global addressing space for
resource and service discovery.
\item \textbf{Uniform Interface:}
Resources are manipulated using a fixed set of four create, read,
update and delete operations: {\itshape PUT, GET, POST} and {\itshape
DELETE}. 
{\itshape PUT} creates a new resource, which can be deleted using
{\itshape DELETE}.
{\itshape GET} retrieves the current state of the resource in some
representation, and {\itshape POST} transfers a new state onto a
resource.
\item \textbf{Self-descriptive messages:}
Resources are decoupled from their representation so that their
content can be accessed in a variety of formats (XML, CSV, JPEG,
etc.).
\item \textbf{Stateful interactions through hyperlinks:}
Every interaction with a resource is stateless, i.e., request messages
are self-contained.
\end{enumerate}

The systems that follow Fielding REST principles are referred to as
\textbf{RESTful}.



% style provides a set of principles that define how Web
%standards, such as HTTP and URIs should be used to exploit Web
%architecture in benefit of the designed system.


%implemented using the {\itshape Dashboard framework} (see appendix
%\ref{ap:dashboardF}), therefore, it takes the benefit of its {\itshape
%client-server} architecture, which follows a {\itshape Model View
%Controller}\cite{bib:MVC} pattern. This pattern isolates business
%logic from user interface, resulting in a application easy to modify
%either the visualization of the application or the underlying layers.



\subsection{AGIS Components}
The three layers group logically the main functionalities of AGIS:
data collection, management of persistent data collected from the
diverse sources, and the data exposing available in AGIS.

\begin{description}
\item[Data Access layer] 
The Data Access layer corresponds to the data
access interface and allows the management of the persistent data,
which is stored in a RDBMS system.
The components of this layer provide functionalities to query and
update the stored data. Access to the database is accomplished by
using a connection pool in order to increase the performance of the
system.
%The storage and web layer are in the server side and they provide
%services for the data collection, storage management and data
%retrieval.

\item[Collectors layer] provides capabilities for the data collection.  The
ATLAS Information System has six different collectors (see
Sec. \ref{sec:dataCollectors}), each one retrieving data from
different sources.  
These collectors perform the service of data retrieving and processing,
giving the proper format to the data, in order to be passed to the
persistent storage (DAO layer).
%The agents collect periodically the data and then

%The storage layer is in charge of the data management. The agents
%collect periodically the data and then using a {\itshape data access
%interface} insert the collected data.


\item[Client connection layer] 
allows heterogeneous ``application-to-application'' and
``user-application'' communication.
The clients may contact the system to retrieve AGIS data using direct
HTTP access, APIs, command lines tools (CLI tools), or the web
interface (for users).



\end{description}



The client connection layer is responsible for providing the
functionality to establish the interaction between
clients and AGIS.
The main components of this layer are:

\begin{description}
\item[Controller:] 
It receives all client requests and keeps the association between the
requested URLs and the corresponding {\itshape actions} executing them
and returning the data in a format requested by the client.
%decides what shoud be done with them.

\item[Action:] 
For each client request there is a corresponding {\itshape action},
which executes a set of operations, which may involve accessing the
database via the DAO layer, allowing to query or update the available
AGIS data.

\item[View:] In general a client request involves output
data. An action puts any collected data in a shared area, so that
this data can be taken by the view to produce the output in a
specific format. This format is identified by its {\itshape
mime/type}, which can be obtained from the header of the HTTP
request.

The Fig. \ref{fig:requestHeader} shows an extract of the header of a
HTTP request to AGIS. 
It is a typical request made by a browser. 
The first line shows the HTTP method used --in this case GET-- and the
resource being accessed --site. 
The next line shows the different output formats of the requested
data.
\end{description}

\begin{figure}[h!tpb]
\begin{Verbatim}[commandchars=\\\�\?,frame=single,fontsize=\relsize{-1}]
GET /dashboard/request.py/site HTTP/1.1
Host: agis.cern.ch
Accept: text/html,application/xhtml+xml,application/xml;q=0.9,*/*;q=0.8
\end{Verbatim}
\label{fig:requestHeader}
\end{figure}







The interaction between the users and applications with AGIS is done
using HTTP requests, so HTTP native features were used as much as
possible in all parts of the implementation of the system.  

AGIS follows the RESTful style, therefore, %it must identify all its
{\itshape resources} --in the RESTful scope-- are identified with an
ID.
These mentioned {\itshape resources} may refer to a 
request for cloud information, a request for site that belongs to a
determined tier, for the space tokens of a determined Storage Element
service, etc. 
AGIS exposes all interesting information as URL identifiable
resources.
The methods for manipulating these resources are standardized HTTP
verbs, which increase process visibility and interoperability.


%Having a single, globally unified naming scheme apply benefits
%specially the machine-to-machine communication.


%As RESTful style proposed, it must be used URIs to identify everything
%that merits being identifiable, specifically, all of the
%``high-level'' resources that the application provides, whether they
%represent individual items, collections of items, virtual and physical
%objects, or computation results.



Therefore, when a client needs to access AGIS data, it must contact
the AGIS services, which are accessed as AGIS resources identified with an
URL   % On the conceptual level a service is a software component
%provided through a network-accessible endpoint.  
%Thus, the clients send HTTP requests to contact the services, which are identified wiht an URL, 
(as the first principle of RESTful proposes).

The AGIS resources are accessed using the standard methods (second
principle of RESTful): PUT, GET, POST and DELETE and they are
decoupled from their representation (third principle of RESTful).
Furthermore, every interaction with a resource is stateless and
context-free, which promotes the scalability through the feasibilitly
of cashing and reduced workload on the server.  

For instance, the AGIS {\itshape cloud resource} is referenced with an
URL. 
A POST request to this resource results in the creation of a new cloud
in the AGIS system. 
A GET request to the cloud resource will return a list of existing
clouds, furthermore
the client can add some parameters in the GET request in order to retrieve
specific resource data. 
The DELETE request eliminates a cloud from the system. \\

%\begin{table}[h!tpb]
%\begin{center}
%\begin{tabularx}{\textwidth}{|>{\hsize=0.2\hsize}X|>{\hsize=0.2\hsize}X|>{\hsize=0.3\hsize}X|>{\hsize=0.3\hsize}X|}
%\hline
%\textbf{Method}  & \textbf{Resource} & \textbf{Description} &\textbf{Output Format}\\ \hline
%GET    &cloud    &display a list of clouds an their properties    &*/*, $application/json$, text/csv, text/html, text/xml\\\hline
%POST   &cloud	 &add a cloud to the system &*/*, application/json, text/html,text/xml \\\hline
%DELETE &cloud	 &delete a  cloud from the system  & \\\hline
%\end{tabularx}
%\end{center}
%\caption{Entities of the ATLAS Grid Information System.}
%\label{tab:cloudResource}
%\end{table}


\section{AGIS Development}

The development of the ATLAS Grid Information System (AGIS) had four main
stages:
\begin{description}
\item[Requirements Collection:] 
In this stage it was identified the clients of the ATLAS Information
System and their requirements.

\item[Identification of Sources of Information:] 
AGIS must provide the static and semi-static information of the
resources, services and topology of the grid, thus, in this stage we
identified all the sources (coming from the three different
sub-grids) that provide that information.

\item[Data Model Design:] 
Once the requirements were collected  
a data model of the ATLAS Information System was defined.

\item[Implementation:] 
This is the programming stage, where the system was implemented,
using the functionalities of Dashboard framework
\cite{monitoringDash}, the Python programming language, the HTTP
server and its {\itshape mod\_python} extension.
\end{description}



The following sections describe in more detail each one of the stages
mentioned before.
\subsection{Requirements Collection}
\label{sec:reqColl}
The ATLAS grid has several components, which need to be aware about
the resources, services and topology of the grid. Various of these
componentes are the clients of AGIS\footnote{ATLAS Grid Information
System
Clients. \url{https://twiki.cern.ch/twiki/bin/view/Atlas/AtlasGIS}},
and they are described as follows.
%(which are described in Sec. \ref{sec:clients}).


%\subsubsection{Clients of the System}

\begin{description}
\item[Data Distributed Management:]
The Distributed Data Management (DDM) is the component in charge to
manage the access to ATLAS experiment data that is distributed at
sites all around the world. DDM is the central link between all the
components of the ATLAS grid, because of data access is needed for any
processing and analysis step (see Section \ref{sec:ddm}).


%Data Distributed Management (DDM) is the area in charge to manage the
%access to ATLAS data as a whole grid, which is distributed at sites
%all around the world.

%DDM is composed by different components which interact together to
%perform different bookeeping and data movement tasks. The catalogues
%are in charge of the bookeeping, while the site services try to answer
%the different requests to spread the ATLAS data among the sites of the
%grid.

%\subsubsection{Central Catalogs}
%The central catalogs define the datasets and their locations. The dataset
%is an aggregation of data (one or more physical file), which are
%processed together and serve collective as input or output of a
%computation or data aquisition process.
%There are four different catalogs:
%\begin{itemize}
%\item Content
%\item Location
%\item Repository
%\item Subscription
%\end{itemize}
%\subsubsection{Site Services}

%The site services are collections of agents, and their principal goal
% is to satisfy the \emph{subscription}. The subscription is the process
%used to move massive amounts of data between sites, on scheduled data
%movements. Subscriptions are typically used to ship data from data
%taking from the Tier-0 to the Tier-1s and Tier-2s. For instance, if
%there is a dataset $A$ presented in a site $Y$ but not in the site $X$,
%and $X$ needs dataset $A$, it must to \emph{subscribe} to dataset
%$A$. Later $A$ is transferred to site $X$ and registered properly in
%catalogs.


\item[Production System:]
The ATLAS Production System (ProdSys) performs the simulation data
production of the experiment using grid resources and provides a
robust framework to execute a large number of jobs in the grid
infrastuctures (see Section \ref{sec:ProdSys}).


\item[PANDA:]
PANDA gives support both for managed production and for user analysis via
flexible job specifications and injection. 
It allows to submit jobs to ATLAS grid independently of the used
middleware (see Section \ref{sec:PANDA}).


\item[Panda Pilot:]
The Panda pilot \cite{pandapilot} is a lightweight execution environment used to prepare
the computing element, request the actual payload (a production or
user analysis job), execute it, and clean up when the payload has
finished. The pilot jobs are broadcasted from the Job Scheduler to the
batch systems and the grid sites. The actual payload is scheduled when
a CPU becomes available, leading to a low latency for analysis
tasks. 
For robustness, the pilot jobs can be submitted either through
Condor-G or locally.


\item[Ganga:] 
Ganga is a frontend tool for job definition and management and mainly
it enables to the users to submit jobs to a variety of
resources (see Section \ref{sec:GANGA}).

\item[Pathena:] Pathena \cite{pathena} is a glue script to submit user-defined jobs to
distributed analysis systems.
It provides a consistent user-interface to Athena users. It allows to
archive user's working directory, send the archive to PANDA, extract
job configuration from jobs, define job specifications automatically
and submit jobs.


\item[Monitoring System:]
It provides a uniform and complete view of various activities like job
processing, data movement and publishing, access to distributed
databases regardless of the underlying multigrid environment (see
Section \ref{sec:mon}).


\item[Accounting System:]
The Accounting System is a component of a grid middleware, which
purpose is to provide an accurate view of the resource and service
usage in the system (see Chapter \ref{accsys}) .


\item[Software Installation:] 
The grid nodes are installed with standard distributions kits and
after installation step, each site is validated using a {\itshape
KitValidation}. 
When a site is validated for a given release number, the relevant tag
is published to the Information System.  

The ATLAS Software Installation System is able of managing user requests
for installation testing and removals of releases.


%\item[Production Shifters]

\item[Santa Claus:]
Santa Claus \cite{santaclaus} is an application that runs one instance for each "kind"
of data (i.e. project) it has to distribute.
Possibly, different instances can be hosted on different hosts. 


\item[NorduGrid:]
The NorduGrid middleware or Advanced Resource Connector (Section
\ref{sec:arc}) is a software solution, enabling production quality
computational and data grids.

\end{description}

%%%%%%%%%%%%%%%%%%%%%%%%%%%%%%%%%%%%%%%%%%%%%%%%%%%%%%%%%%%%%%%%%%%%%
%\begin{com}
%THIS IS JUST AN ENUMERATION OF SUBSECTIONS (AT PRESENT)!!!
%IT SHOULD BE COMPLETED!!!
%\end{com}
%%%%%%%%%%%%%%%%%%%%%%%%%%%%%%%%%%%%%%%%%%%%%%%%%%%%%%%%%%%%%%%%%%%%%



\subsubsection{Requirements}

In this stage were collected several requirements, which were
collected from the clients previously described and they were
classified in AGIS {\itshape entities} and {\itshape
functionalities}.

The Table \ref{tab:entities} describes the main entities that were identified
and the Table
\ref{tab:functionality} summarizes the AGIS functionalities.

\begin{table}[h!tpb]
\begin{center}
\begin{tabularx}{\textwidth}{|>{\hsize=0.3\hsize}X||>{\hsize=0.7\hsize}X|}
\hline

\textbf{Entities} & \textbf{Description} \\ \hline\hline

Site & Entity that aggregates a set of services and resources\\\hline

Cloud & Entity that aggregates set of sites that belong to different
tiers, this takes importance in the data distribution \\\hline

Region & Sites belong to geographical regions \\\hline

Country & Sites belong to a country\\\hline

Grid & The grid infrastructure (EGEE, OSG or NDGF) \\\hline

Tier& Entity that groups sites, which satisfy some minimal
characteristic, such as storage capacity, CPUs and network
bandwith.\\\hline

GOCDB name& The official name of a site, provided by GOCDB\\\hline

Domain& Attribute associated to a site, and it is the base name that
groups a set of resources \\\hline

Contact& Attribute associated to a site, it corresponds to the
email of the administrator of the site\\\hline

Alias& Attribute associated to a site, because a site can have a
different name for a particular service. \\\hline

%Alias for group of sites& \\

Federated Sites& Attribute associated to a site. The site can belong
be part of a federated site (NDGF sites).\\\hline

Local Catalog &  Service entity that is involved in the data management\\\hline

Storages Element & Service entity that is involved with the data
management providing storage space \\\hline

Space Tokens & Entity that corresponds to a storage space, and can
have its own set of policies \\\hline

Computing Elements& Service entity that provide functionalities to
send the jobs to the grid queues \\\hline

Queue&  Entity that run the jobs \\\hline

FTS& Service entity that allows the data movement into the grid resources \\\hline

%Software Version & \\\hline
\end{tabularx}
\end{center}
\caption{AGIS Entities}
\label{tab:entities}
\end{table}

\begin{table}[h!tpb]
\begin{center}
\begin{tabularx}{\textwidth}{|>{\hsize=0.3\hsize}X||>{\hsize=0.7\hsize}X|}
\hline

\textbf{Functionality} & \textbf{Description} \\ \hline\hline

Get entity & It retrieves ATLAS entities (cloud, sites, services,
etc.)\\\hline

Get entity properties & It retrieves attributes associated to the
entities \\\hline

Get associated service & It retrieves services associated to some
entity \\\hline

Provide specific functions for each entity &For instance, ``Get close
sites'', which is a DDM specific functionality that retrieves  the
sites belonging to a cloud, for a given site, or ``renaming of sites'', which
provides the functionality of renaming a site in a easy way.\\\hline

Get topology & It retrieves all the ATLAS entities with their
associations and attributes.\\\hline

Provide APIs & The system will be used by ATLAS components and users,
therefore, it must provides the necessary APIs or tools to access the
system. \\\hline

Secure Access & The data modification in the system must be controlled
with some security mechanism. \\\hline

Provide command lines tools & These tools should be used by different
users through the CERN lxplus terminals. \\\hline

Provide Web Application & To visualize in an easy way the complete
ATLAS topology, services and resources. \\\hline

Direct HTTP access & To provide an easy access to the AGIS services. \\\hline
%SURL's site or cloud & It retrieves the SURL associated to a site or
%cloud\\\hline

%Get all sources  & \\\hline


%Get space tokens &  \\\hline
%Get all destinations  & \\\hline
%Get I/O protocols & \\\hline
\end{tabularx}
\end{center}
\caption{AGIS Functionalities Summary}
\label{tab:functionality}
\end{table}

In addition to the functional requirements are nonfunctional
requirements, but they are described and analysed in Chapter
\ref{performance}.

\subsection{Identification of Sources of Information}
\label{sec:sourcesInfo}
%In the Collection Requirement stage where identified the entities and
%the functionality that should provide the ATLAS Grid Information
%System. 

The ATLAS grid is composed for three different sub-grids, which
implies there are several sources of information that provides data
related with the resources, services and topology of the ATLAS grid.

Once the {\itshape Requirements Collection} was done, it was possible
to identify the following sources of information (see
Sec. \ref{sec:sources} for more details):
\begin{itemize}
\item TiersOfATLAS: It provides information related with topology and
{\itshape Storage Element} service data.

\item GOCDB: It provides the official list of the sites --with their
associated attributes-- belonging to the EGEE and NDGF sub-grids.

\item OIM: It provides the official list of the sites --with their
associated attributes-- belonging to the OSG sub-grid.

\item Panda DB: It provides information related with the {\itshape
queues} and {\itshape panda queues} associated to the {\itshape
Computing Element} service

\item BDII: It provides information related with the services of the
ATLAS grid (Storage Element, Computing Element, File Transfer Service,
Local Catalog Service, Proxy Service)

\item NDGF: It provides information related with the sites --with their
associated attributes-- belonging to the NDGF sub-grid.

\end{itemize}


%The Fig. \ref{fig:sourcesOfInf} depicts the different sources of
%information and the data retrieved from each one.

%\begin{figure}[!htbp]
%\centering
%%\epsfig{file=image/sourcesOfInf.eps,scale=0.55}
%\caption{Sources of Information}
%\label{fig:sourcesOfInf}
%\end{figure}

\subsection{Data Model Design}
One of the most important aspect in the data model of the ATLAS Grid
Information System, is that it must represent the ATLAS grid topology,
because currently there is not any system that can provide it. The
main topology entities identified were: {\itshape site, cloud, tier}
and {\itshape services}. Fig. \ref{fig:dataModel} shows the {\itshape
entity-relationship} diagram.

\begin{figure}[!htbp]
\centering
\pgfuseimage{databaseSchemaAGIS}
%\epsfig{file=images/dataModel.eps,scale=0.55}
\caption{AGIS Data Model Overview}
\label{fig:dataModel}
\end{figure}

\newpage
\subsection{Implementation}

The ATLAS Grid Information System was implemented using the
functionalities provided by the {\itshape Dashboard Framework}. Python is the programming language used
to implement the system.

The implementation had different stages, the development of Data
Access Object (DAO), the APIs, the data collectors, the command
lines (CLI) tools and the web interface, which are described as
follows.

\subsubsection{Data Access Object}
The implementation of  a Data Access Object (DAO) allows to have a
clear design and maintainability of the system, because	 the
application queries are decoupled from the internal implementation of
the data storage.

The DAO represents the {\itshape data access interface}, which are a
public set of methods for the update and retrieval of information. 


%\begin{figure}[h!tpb]
%\centering
%\epsfig{file=images/agisDAO.eps,scale=0.55}
%\caption{Database access}
%\label{fig:agisDAO}
%\end{figure}

The {\itshape connection pool} is the method used to access  the
database, in order to reduce the overhead in creating new connections,
reducing the load on the server and increasing the performance.




\subsubsection{APIs}%\footnote{\textbf{A}pplication \textbf{P}rogramming
%\textbf{I}nterfaces} 

{\itshape AGISQuery} and {\itshape AGISUpdate} are the two APIs
developed to query and update, respectively, the ATLAS Grid
Information System. 


The {\itshape AGISQuery} API provides different methods to query the
ATLAS Grid Information System. The table \ref{tab:agisQueryMethods}
describe these methods.



\begin{table}[h!tpb]
\begin{center}
\begin{tabularx}{410pt}{>{\hsize=0.3\hsize}X>{\hsize=0.8\hsize}X}
\hline


\textbf{Method} & \textbf{Description} 

\\ \hline

getTopology & It retrieves all the ATLAS grid entities with their
properties and associations\\\hline

listClouds & It retrieves all the clouds of the ATLAS grid with all
their attributes\\\hline

listSites & It retrieves all the sites of the ATLAS grid with all
their attributes\\\hline

listServices &It retrieves all the services belonging to the sites of
the ATLAS grid, with all their attributes \\\hline

listSpaceTokens & It retrieves all the space tokens belonging to the {\itshape Storage Elements} services of the ATLAS grid, with all their attributes\\\hline

listQueues & It retrieves all the queues belonging to the  {\itshape Computing Elements} services of the ATLAS grid, with all their attributes\\\hline

getTiersOfAtlas & It retrieves all the ATLAS grid entities with their properties and associations in the {TiersOfATLAS} style\\\hline
\end{tabularx}
\end{center}
\caption{{\itshape AGISQuery} API methods}
\label{tab:agisQueryMethods}
\end{table}


The {\itshape AGISUpdate} API allows to update the data of the ATLAS
Grid Information System. The table \ref{tab:agisUpdateMethods} shows
the methods of this API.

\begin{table}[h!tpb]
\begin{center}
\begin{tabularx}{410pt}{>{\hsize=0.3\hsize}X>{\hsize=0.8\hsize}X}
\hline

\textbf{Method} & \textbf{Description} 

\\ \hline
addCloud & It inserts an ATLAS  cloud to the system\\\hline
removeCloud & It removes an ATLAS cloud from the system\\\hline
addSite &It inserts an ATLAS site to the system \\\hline
removeSite & It deletes an ATLAS site from the system\\\hline
addService & It inserts an ATLAS service to the sytem\\\hline
removeService & It deletes an ATLAS service from the system\\\hline

\end{tabularx}
\end{center}
\caption{{\itshape AGISUpdate API methods}}
\label{tab:agisUpdateMethods}
\end{table}

To update the system it is necessary to have some permissions. The
authentication is done via X509 proxy certificates.

The APIs provided by the ATLAS Grid Information System works using
{\itshape HTTP requests}. The following code is an extract
of the {\itshape AGISUpdate} API code. In (1) {\itshape client} is an
instance of the {\itshape URLClient}\footnote{URLClient is based on
pycurl, \url{http://pycurl.sourceforge.net/doc/pycurl.html}}, which is
a specialized object that allows to transfer data from or to a server,
using HTTP protocol. Next line (2) returns the full endpoint of the
service --in this case, the service that allows to remove the cloud
entity from the system--. Finally in (3) it performs the HTTP DELETE
request on the specific URL (the endpoint) and the specific resource
(the cloud).

\newpage
\begin{Verbatim}[commandchars=\\\�\?,frame=single,fontsize=\relsize{-1}]
client = URLClient(secure=True) (1)
endpoint = self._getEndpoint(self.hostname, self.port, "cloud", secure=True)
client.httpDelete(endpoint, params = cloud_name, self.outputFormat = None)
\end{Verbatim}




The APIs are available for CERN users through the
AFS\footnote{\textbf{A}ndrew \textbf{F}ile \textbf{S}ystem,
distributed networked file system which uses a set of trusted servers
to present a homogeneous, location-transparent file name space to all
the client workstations.}. The users can call to a running instance of
the ATLAS Grid Information System server. The python code below  shows one of the ways to use the APIs, in
this case importing the modules in a {\itshape python shell}.


\begin{Verbatim}[commandchars=\\\�\?,frame=single,fontsize=\relsize{-1}]
user@host:\~\$ python
Python 2.5.2 (r252:60911, Apr 17 2008, 13:15:05)
Type "help", "copyright", "credits" or "license" for more information.
>>> from dashboard.api.agis.AGISQuery import AGISQuery
>>> agisQuery = AGISQuery(hostname="agis.cern.ch", port=80)
>>> agisQuery.listClouds()
\end{Verbatim}


\subsubsection{Data Collectors}
\label{sec:dataCollectors}

The data collection is done using services that retrieves periodically
data from the different sources of information.  These collector
services are daemons that can contact diverse hosts and collect
information of interest. The implementation is based on the
{\itshape arda.dashboard.service-config} module provided by the Dashboard
Framework \cite{monitoringDash}.

There are six different sources of information, thus six different
services that retrieve the data and store it into a local database.
\begin{description}
\item {\itshape TiersOfATLASCollector:} This collector retrieves data from
the TiersOfATLAS source. Firstly, this collector have to fetch the
{\itshape TiersOfATLASCache.py} file from an URL and then have to
process this file to obtain information related principally with the
ATLAS grid topology. TiersOfATLAS is the only source that gives
information of the main entities of the ATLAS topology --tiers and
clouds--. This collector also retrieves data related with the {\itshape
space tokens} associated to the {\itshape Storage Elements} services.
\item {\itshape GOCDBCollector:} The Grid Operation Center Data Base
(GOCDB) is the official source that provides the full list of sites
belonging to the ATLAS grid (except the OSG sites), thus, this
collector retrieves periodically all the site information.
\item {\itshape OIMCollector}: This collector retrieves the official list
of sites belonging to the OSG sub-grid.
\item {\itshape BDIICollector:} This collector retrieves data related with
all the services of the ATLAS grid.
\item {\itshape PandaCollector}: This collector retrieves data related
with the {\itshape panda queues} associated to the {\itshape Computing
Element} services.
\item {\itshape NDGFCollector}: This collector retrieves all the data related with the NDGF sites and services.
\end{description}

The data collection implies regular access to the information
sources. To provide a reliable ATLAS Grid Information System, the
collector services should run constantly and need to recover any
missing data in case of eventual failure and following restart.  The
functionalities of the Dashboard Framework allows to monitor the
collectors and to be aware about the status of them, using some
reporting methods: a web application showing the status of the
collectors, an email or a SMS.



%\newpage
\subsubsection{CLI Tools}
The CLI tools allows to the users to update and retrieve information
to or from the ATLAS Grid Information System, using command lines.

The set of available tools and their functionality is presented in
Table \ref{tab:cliTools}

\begin{table}[h!tpb]
\begin{center}
\begin{tabularx}{350pt}{>{\hsize=0.3\hsize}X||>{\hsize=0.8\hsize}X}
\hline


\textbf{CLI Tool Name} & \textbf{Functionality} \\ \hline\hline
agis-clouds	 & Lists ATLAS clouds stored in the AGIS system.\\\hline

agis-sites	& Lists ATLAS sites stored in the AGIS system.\\\hline

agis-services	& Lists ATLAS services stored in the AGIS system.\\\hline

agis-spaceTokens &Lists ATLAS spacetokens stored in the AGIS system.\\\hline

agis-queues	& Lists ATLAS queues stored in the AGIS system.\\\hline
\end{tabularx}
\end{center}
\caption{CLI Tools}
\label{tab:cliTools}
\end{table}

The class that implements a CLI tool calls the proper API ({\itshape
AGISQuery} or {\itshape AGISUpdate}) to perform the requested
action (see Fig. \ref{fig:CLIcode}). As it was mentioned before, the API performs HTTP requests to
retrieve the data.

\begin{figure}[h!tpb]
%\begin{Verbatim}[commandchars=\\\�\?,frame=single,fontsize=\relsize{-1}]
\begin{lstlisting}[language=Python,basicstyle=\footnotesize]
def execute(self):
        # Perform the remote call
        try:
            agisQuery = AGISQuery(hostname=self.options.service, 
	    port=self.options.port, outputFormat=self.options.format)
            clouds = agisQuery.listClouds(pair=self.options.pair)
        except DashboardException, exc:
            self._logger.error(exc)
            sys.exit(-1)
\end{lstlisting}
%\end{Verbatim}
\caption{CLI Tool data request.}
\label{fig:CLIcode}
\end{figure}

\newpage
\subsubsection{Web Interface}

The web interface allows to visualize and update the AGIS data in an
easy way.

%The web interface navigation is usually done as presented in
%Fig. \ref{fig:webNavigation}.

%\begin{figure}[h!tpb]
%\centering
%\epsfig{file=images/webNavigation.eps,scale=0.55}
%\caption{Database access}
%\label{fig:webNavigation}
%\end{figure}

The web interface provides web pages, which are structured in the main
 components described below:

\begin{description}
\item {\itshape Topology:} This component allows to visualize all the
entities belonging to the ATLAS topology, in a {\itshape top-down}
approach, that is, firstly it is possible to see an overview of the
ATLAS topology --{\itshape the clouds}-- --{\itshape the clouds}--,
and then it is possible to go deep, to go finally to the base elements
of the topology --{\itshape queues} and {\itshape space tokens}.

\item {\itshape Clouds:} This component allows to
visualize and to manage all the \-ATLAS\- clouds. It is possible to add,
remove and update clouds, for wich, the authentication is done using
X509 proxy certificates.

\item {\itshape Sites:} This component allows to visualize all the
sites, together their main attributes (cloud, tier, country, alias,
region, federated site, contact and source of information) belonging
to the ATLAS grid, and to add, remove or update the site
information. The authentication is done using X509 proxy certificates.

\item {\itshape Services:} This component allows to visualize all the
services belonging to the ATLAS grid, which are described by their
{\itshape endpoint, type of service} and the site where they
belong. Also, it is possible to to add, remove or update the site
information. The authentication is done using X509 proxy certificates.

\item {\itshape Blacklisting:} This component lists all the
blacklistings scheduled in the ATLAS grid, and to add or remove
blacklistings. The authentication is done using X509 proxy
certificates.
\end{description}

As it was mentioned before, all the processes of adding, removing or
updating information are done with authentication, using X509 proxy
certificates. These certificates are commonly used in security systems
\cite{bib:X509Proxy}  are a standard for dynamic
delegation and identity creation in public key infrastructures.

%\begin{figure}[h!tpb]
%\centering
%\epsfig{file=images/webOverview.eps,scale=0.55}
%\caption{Web Interface Overview.}
%\label{fig:webOverview}
%\end{figure}

%\subsubsection{Sources of Information}
%\begin{itemize}
%\item Sources: TiersOfAtlas, BDII \ldots
%\item Data Model: explain why to choose a relational database and not
%  a LDAP model,or vice versa \ldots
%\item Database: Oracle, use database index \ldots
%\end{itemize}
%\subsection{Results}


%%%%%%%%%%%%%%%%%%%%%%%%%%%%%%%%%%%%%%%%%%%%%%%%%%%%%%%%%%%%%%%%%%%%%
%\begin{com}
%ESTO ES UN ESQUELETO QUE, SUPONGO, PRONTO REVESTIRAS DE MUSCULOS,
%ARTERIAS, ETC.
%¿PERO CUAL ES LA IDEA GENERAL AQUI?
%\end{com}
%%%%%%%%%%%%%%%%%%%%%%%%%%%%%%%%%%%%%%%%%%%%%%%%%%%%%%%%%%%%%%%%%%%%%


%\chapter{Discovery Mechanisms}

%One important issue in GIS is the discovery of the resources at the
%grid system. The discovery in distributed system is an area that have
%been studied \ldots
